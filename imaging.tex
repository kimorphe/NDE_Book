%%%%%%%%%%%%%%%%%%%%%%%%%%%%%%%%%%%%%%%%%%%%%
\section{形状再構成問題の数値計算例}
\subsection{はじめに}
本節では、有限な大きさの散乱体を対象として,これまでに示した
方法に基づいて散乱体の形状再構成を行う.
散乱体は円形空洞と割れの2つをとりあげ,いくつかの異なる
画像化式によってイメージングを行い結果を比較する.
これらのイメージングには,数値シミュレーションによって
合成した散乱波の時間波形を用いる.
イメージングは開口合成法の一種とみなすことのできる
画像化関数を用い,全て時間領域における計算によって行う.
\subsection{問題設定}
図\ref{fig:imaging_problem}に,2次元イメージングを行う
2つのモデルを示す.この図の(a)は円形空洞を,(b)は滑らかに曲がった
割れを含む半無限領域を示している.ここでは,半無限領域の表面に設置した
リニアアレイ探触子から面外波(SH波)を送受信して散乱波形を取得し,
これら2つの欠陥について形状再構成(イメージング)を行う.
その際,送受信は全素子の組み合わせで行い,得られた波形の一部あるいは全部
を使ってイメージングを行う.
なお,図\ref{fig:imaging_problem}の寸法は,後にも述べる通り,
送信波の波長で無次元化したものである.また,図中の破線で示した範囲は画像化を行う領域を表す.
図\ref{fig:linear_array}はリニアアレイ探触子の素子配置を示す.この図では,アレイ探触子を構成する
素子数を$M$とし,半無限媒体の表面において,第$i$番目のアレイ素子が占める箇所を$e_i$として示している.
なお,各素子の幅$w$と,隣接する素子の間隔であるピッチ$p$は
一定としている.このとき,アレイ探触子の開口(aperture)$a_p$は
次式で与えられる.
\begin{equation}
	a_p=p(M-1)+w
	\label{eqn:aperture}
\end{equation}
図\ref{fig:imaging_problem}では,これらリニアアレイ探触子の配置を
\[
	p=1, w=0.5, M=32, (a_p=32)
\]
とし,アレイ探触子の中心が欠陥の直上に来るようにする.
図\ref{fig:imaging_problem}は,このようにして配置したアレイ素子の位置を,
媒体表面($x_2=30$)上の黒の点で表している.
\begin{figure}[h]
	\begin{center}
	\includegraphics[width=1.0\linewidth]{Figs/model.pdf} 
	\end{center}
	\caption{ 形状再構成問題のための数値計算モデル.}
	\label{fig:imging_problem}
\end{figure}
\begin{figure}[h]
	\begin{center}
	\includegraphics[width=0.6\linewidth]{Figs/array.pdf} 
	\end{center}
	\caption{リニアアレイセンサーを模擬した送受信領域の設定.} 
	\label{fig:linear_array}
\end{figure}
各アレイ素子は,送信時には予め指定された面外方向表面力$\bar{t}_3(t)$を,
素子範囲$e_j$内で一様に加えて面外波を励起する.
受信センサーとしては,面外方向の粒子速度を観測するものとする.
このとき,素子$i$で送信し,素子$j$で受信された波形$a_{ij}(t)$は,
素子$j$の範囲$e_j$で生じる速度場を次のように平均化したものとして与える.
\begin{equation}
	a_{ij}(t)=\frac{1}{w}\int_{e_j} v_3(\fat{x},t)ds, \ \ (i,j=1,2,\dots M)
	\label{eqn:WWW_001}
\end{equation}
ただし$v_3(\fat{x},t)$は,$(\fat{x},t)$における面外方向粒子速度を表す.
いま,送信素子の番号を$i$,送信開始時刻を$t=0$とし,それ以前には媒体は
静止した状態にあるとする.
半無限領域を$B$,その中で欠陥が占める領域を$D$とすれば,
面外波動場の支配方程式系は,以下のように表される.
\begin{equation}
	\rho \dot v_3 = \sigma_{31,1}+\sigma_{32,2}, 
	\ \ (\fat{x} \in B \setminus \bar{D},\, t>0)
	\label{eqn:WWW_002}
\end{equation}
\begin{equation}
	\sigma_{3\alpha}=\mu u_{3,\alpha}, 
	\ \ (\fat{x} \in B \setminus \bar{D},\, t>0)
	\label{eqn:WWW_003}
\end{equation}
\begin{equation}
	u_3(\fat{x},0)=0, v_3(\fat{x},0)=0, \ \ (\fat{x} \in B \setminus \bar{D})
	\label{eqn:WWW_003}
\end{equation}
\begin{equation}
	t_3^{(n)}(\fat{x},t)=
	\left\{
	\begin{array}{cc}
		0, & \left(\fat{x}\in (\partial H \setminus e_i ) \cup \partial D,\,  t>0 \right) \\
		\bar{t}_3(t), & \left(\fat{x}\in (\partial H\setminus e_i)  \cup \partial D,\, t>0 \right)\\
	\end{array}
	\right.
	\label{eqn:WWW_004}
\end{equation}
ただし,$\fat{x}=(x_1,x_2)$,$u_3(\fat{x},t)$は面外変位を,$\mu$はせん断剛性を
表し,$t^{(n)}_3$は,
\begin{equation}
	t^{(n)}_3= \sigma_{3\beta}n_\beta
	\label{eqn:WWW_004}
\end{equation}
で与えられる表面力ベクトルの面外方向成分である.なお,2次元問題のため,指標$\alpha,\beta$は
1または2をとる.
ここでは,送信波形を次のようにRicker波で与える.
\begin{equation}
	\bar{t}_3(t)=\left\{ 1-2\left(\pi f_0t\right)^2 \right\}\exp\left\{-(\pi f_0 t)^2\right\}
	\label{eqn:WWW_005}
\end{equation}
なお,$f_0$はピーク周波数を表し,式(\ref{eqn:WWW_005})のRicker波とその周波数スペクトルを
描くと図\ref{fig:Ricker}のようになる.
\begin{figure}[h]
	\begin{center}
	\includegraphics[width=0.8\linewidth]{Figs/ricker.pdf} 
	\end{center}
	\caption{入射波を励起するトラクション波形に用いる(a)Ricker波と
	(b)その周波数スペクトル.} 
	\label{fig:ricker}
\end{figure}
横波の位相速度$c_T$は,密度$\rho$とせん断剛性$\mu$を用いて,$c_T=\sqrt{\frac{\mu}{\rho}}$
で与えられ,周波数$f_0$における波長は
\begin{equation}
	\lambda_0 = c_T/f_0
	\label{eqn:WWW_006}
\end{equation}
で得られる.本節では,長さに関する量は波長$\lambda_0$で,時間に関する量は
周期$T_0=1/f_0$規格化して示す.このことは,図\ref{fig:imaging_problem}に
示した領域の大きさについても同様である.
%%%%%%%%%%%%%%%%%%%%%%%%%%%%%%%%%%%%%%%%
\subsection{散乱波形の合成 (FDTD法)}
イメージングに用いる波形$\left\{ a_{ij}(t)\left|i,j=1,2,\dots M\right.\right\}$は,
式(\ref{eqn:WWW_002})-(\ref{eqn:WWW_004})の初期値−境界値問題を
FDTD(finite-difference time-domain)法で解き,
式(\ref{eqn:WWW_001})に従って合成する.
FDTD法は有限差分法の一種で,構造格子上で方程式の離散化を行うため,
領域形状を正確に表現することはできない.その一方で,有限要素法や
境界要素法に比べて実装は簡単であり,計算効率も高い.
そのため,超音波探傷試験のシミュレーションのように,あまり高い
計算精度が要求されない問題には非常に使いやすい.
従って超音波イメージングの数値シミュレーションを目的とした
波形合成に有用であるため,以下,均質媒体における2次元面外波について,
FDTD法による離散化についての基本的な事柄を述べておく.

FDTD法では,式(\ref{eqn:WWW_002})と,式(\ref{eqn:WWW_003})の両辺を時間で微分して
得られる
\begin{equation}
	\dot \sigma_{3\alpha}=\mu v_{3,\alpha}
	\label{eqn:WWW_007}
\end{equation}
を連立した,1階の偏微分方程式系を$v_3,\sigma_{31}$および$\sigma_{32}$を
求めるべき未知の波動場として離散化する.
その際,時間に関してはリープフロッグ法を適用し,空間に関しては
スタガード格子を用い,いずれも中央差分で偏微分を近似する.

%\subsection{時間に関する離散化}
\subsection{時間微分の差分近似}
時間の関数$y=f(t)$を中央差分で離散化する.
そのために,時間ステップ長を$\Delta t$, $t=0$を基準に,第$k$番目の時間ステップ
を$t^k=k\Delta t$とする.図\ref{fig:leapfrog}に示すように,
$t=t^k$における$f(t)$の値を$f(t^{k})=f^k$とすれば,
半整数時間ステップ$t^{k+\frac{1}{2}}=(k+\frac{1}{2})\Delta t$における
$f(t)$の微分は$\dot f(t^{k+\frac{1}{2}})$は,中央差分近似により
\begin{equation}
	\dot{f}(t^{k+\frac{1}{2}})
	=\dot{f}^{k+\frac{1}{2}} \approx \frac{ f^{k+1}-f^k}{\Delta t}
	\label{eqn:WWW_008}
\end{equation}
とすることができる.
従って,式(\ref{eqn:WWW_008})を式(\ref{eqn:WWW_002})と式(\ref{eqn:WWW_007})に
あてはめると,
\begin{equation}
	\rho \frac{ (v_3)^{l+1}-(v_3)^l}{\Delta t}
	=
	(\sigma_{31,1})^{l+\frac{1}{2}}
	+
	(\sigma_{31,2})^{l+\frac{1}{2}}, 
	\ \ (k=0,\pm 1, \pm 2\dots )
	\label{eqn:WWW_009}
\end{equation}
と,
\begin{equation}
	\frac{ (\sigma_{3\alpha})^{k+1}-(\sigma_{3\alpha})^k}{\Delta t}
	=
	\mu (v_{3,\alpha})^{k+\frac{1}{2}},
	\ \ (k=0,\pm 1, \pm 2\dots )
	\label{eqn:WWW_010}
\end{equation}
を得る.ただしこれらの式では,近似であることを意味する$\simeq$ではなく
通常の等号($=$)を用いている.
式(\ref{eqn:WWW_009})と(\ref{eqn:WWW_010})を変形すると,
\begin{eqnarray}
	(v_3)^{l+1} &=& 
	(v_3)^l
	+
	\frac{\rho}{\Delta t} \left\{
	(\sigma_{31,1})^{l+\frac{1}{2}} + (\sigma_{31,2})^{l+\frac{1}{2}}
	\right\} 
	\label{eqn:WWW_011}
	\\
	(\sigma_{3\alpha})^{k+1} &=& 
	(\sigma_{3\alpha})^{k} 
	+
	\mu \Delta t (v_{3,\alpha})^{k+\frac{1}{2}}
	\label{eqn:WWW_012}
\end{eqnarray}
となるので,速度$v_3$に関する時間微分を近似するための時間ステップ$l$と,
応力$\sigma_{3\alpha}$のための時間ステップ$k$を
\begin{equation}
	l=k-\frac{1}{2} \Rightarrow t^k=t^l+\frac{\Delta t}{2}
	\label{eqn:WWW_013}
\end{equation}
と,互いに$\Delta t/2$だけずらして設定すれば,式(\ref{eqn:WWW_011})は
\begin{equation}
	(v_3)^{k+\frac{1}{2}} = (v_3)^{k-\frac{1}{2}}
	+
	\frac{\rho}{\Delta t} \left\{
	(\sigma_{31,1})^{k} + (\sigma_{31,2})^{k}
	\right\} 
	\label{eqn:WWW_014}
\end{equation}
となる.
従って,整数時間ステップ$t^k$での応力を用いて,半整数時間ステップ$t^{k+\frac{1}{2}}$
における速度が求められる.一方,式(\ref{eqn:WWW_012})では,
$t^{k+\frac{1}{2}}$における速度から$t^{k+1}$における応力が求められるので,
式(\ref{eqn:WWW_012})と式(\ref{eqn:WWW_014})を交互に用いることで,
応力と速度を順次初期値から更新して求められることが分かる.
このような時間ステッピングの方法をリープフロッグ法と呼ぶ.
\begin{figure}[h]
	\begin{center}
	\includegraphics[width=0.6\linewidth]{Figs/FDgrid_t.pdf} 
	\end{center}
	\caption{リープフロッグ法による時間に関する離散化.} 
	\label{fig:leapfrog}
\end{figure}
%%%%%%%%%%%%%%%%%%%%%%%%%5%
\subsection{空間微分の差分近似}
式(\ref{eqn:WWW_012})と式(\ref{eqn:WWW_014})の右辺には空間に関する微分
\[
	v_{3,1}=\frac{\partial v_3}{\partial x_1},\,
	v_{3,2}=\frac{\partial v_3}{\partial x_2}, \,
	\sigma_{31,1}=\frac{\partial \sigma_{31}}{\partial x_1},\,
	\sigma_{32,2}=\frac{\partial \sigma_{32}}{\partial x_2}
\]
が含まれる.これらの微分を差分近似するために,図\ref{fig:staggered}
の中央に示すように,空間を幅$\Delta x_1$,高さ$\Delta x_2$の
セルに区切り,媒体が占める領域$V$をこれら小さなセルの集まりで近似する.
セルの頂点は,原点から数えて$x_1$方向へ$i$番目,$x_2$方向へ
$j$番目の位置にあるものを,インデックスの対$(i,j)$で参照し,
その位置を
\[
	(x_1)_i=i\Delta x_1, \, (x_2)_j=j\Delta x_2, 
\]
と表す.さらに,位置の関数$f(x_1, x_2)$の
$(i,j)$における値を
\[
	f_{i,j}=f\left((x_1)_i, (x_2)_j\right)
\]
と,下付きのインデックスを用いて書くことにする.
同様にして,セル中央の点を
\[
	\left( 
	(x_1)_{i+\frac{1}{2},j+\frac{1}{2}}, (x_2)_{i+\frac{1}{2},j+\frac{1}{2}}
	\right)
	=
	\left(
	\left(i+\frac{1}{2} \right)\Delta x_1, 
	\left(j+\frac{1}{2} \right)\Delta x_2, 
	\right)
\]
として,この位置における$f(x_1,x_2)$を$f_{i+\frac{1}{2},j+\frac{1}{2}}$と表す.
さらに,時間に関する離散化についての表記を踏まえ,
時間と空間の双方について離散的な点での値を示す際には,
時間ステップを上付きの,空間位置を下付きの添字を用いて
\[
	f\left((x_1)_i,(x_2)_j,t^k\right) =f^k_{i,j}
\]
などと表す.

ここで,速度$v_3$を評価するための点をセル中央にとれば,
式(\ref{eqn:WWW_014})は,$(i+\frac{1}{2}, j+\frac{1}{2})$の点
で評価されることとなるため,この式は
\begin{equation}
	(v_3)_{i+\frac{1}{2},j+\frac{1}{2}}^{k+\frac{1}{2}} 
	= 
	(v_3)_{i-\frac{1}{2},j+\frac{1}{2}}^{k-\frac{1}{2}} 
	+
	\frac{\rho}{\Delta t} \left\{
		(\sigma_{31,1})_{i+\frac{1}{2},j+\frac{1}{2}}^{k} 
		+
		(\sigma_{32,2})_{i+\frac{1}{2},j+\frac{1}{2}}^{k} 
	\right\} 
	\label{eqn:WWW_015}
\end{equation}
として用いることになる.
従って、空間微分$\sigma_{31,1}$と$\sigma_{32,2}$は,セル中央で
評価する必要がある.これらの微分を中央差分で近似するためには,
$\sigma_{31}$の評価点はセル左右の返に位置する$(i,j+\frac{1}{2})$に,
$\sigma_{32}$はセル上下の返に位置する$(i+\frac{1}{2},j)$にとり,
\begin{eqnarray}
	( \sigma_{31,1})_{i+\frac{1}{2}, j+\frac{1}{2}} 
	&\approx& 
	\frac{(\sigma_{31})_{i+1,j+\frac{1}{2}} - (\sigma_{31})_{i,j+\frac{1}{2}} }{\Delta x_1}
	\label{eqn:WWW_016}
	\\
	( \sigma_{32,2})_{i+\frac{1}{2}, j+\frac{1}{2}} 
	&\approx& 
	\frac{(\sigma_{32})_{i+\frac{1}{2},j+1} - (\sigma_{32})_{i+\frac{1}{2},j} }{\Delta x_2}
	\label{eqn:WWW_017}
\end{eqnarray}
とすればよい.
このとき,式(\ref{eqn:WWW_012})は,$\sigma_{3\alpha}$の評価点の位置を考慮すれば
\begin{eqnarray}
	(\sigma_{31})^{k+1}_{i,j+\frac{1}{2}} &=& 
	(\sigma_{31})^{k} _{i,j+\frac{1}{2}}
	+
	\mu \Delta t (v_{3,1})^{k+\frac{1}{2}}_{i,j+\frac{1}{2}}
	\label{eqn:WWW_018}
	\\
	(\sigma_{32})^{k+1}_{i+\frac{1}{2},j} &=& 
	(\sigma_{32})^{k}_{i+\frac{1}{2},j} 
	+
	\mu \Delta t (v_{3,2})^{k+\frac{1}{2}}_{i+\frac{1}{2},j}
	\label{eqn:WWW_019}
\end{eqnarray}
と考えることになる.
これらの式に現れる$v_3$に関する微分は,セル中央での値を用いて次のように中央差分で
近似できる.
\begin{eqnarray}
	(v_{3,1})_{i,j+\frac{1}{2}}
	&\approx & 
	\frac{
		(v_3)_{i+\frac{1}{2},j+\frac{1}{2}} 
		-
		(v_3)_{i-\frac{1}{2},j+\frac{1}{2}} 
	}{\Delta x_1}
	\label{eqn:WWW_020}
	\\
	(v_{3,2})_{i+\frac{1}{2},j}
	&\approx & 
	\frac{
		(v_3)_{i+\frac{1}{2},j+\frac{1}{2}} 
		-
		(v_3)_{i+\frac{1}{2},j-\frac{1}{2}} 
	}{\Delta x_2}
	\label{eqn:WWW_021}
\end{eqnarray}

以上より,式(\ref{eqn:WWW_016}),(\ref{eqn:WWW_017})を式(\ref{eqn:WWW_014})に,
式(\ref{eqn:WWW_020}),(\ref{eqn:WWW_021})を式(\ref{eqn:WWW_012})にそれぞれ代入すれば,
以下のような離散化された方程式系が得られる.
\begin{eqnarray}
	(\sigma_{31})^{k+1}_{i,j+\frac{1}{2}} &=& 
	(\sigma_{31})^{k} _{i,j+\frac{1}{2}}
	+
	\mu \Delta t 
	\frac{
		(v_3)_{i+\frac{1}{2},j+\frac{1}{2}}^{k+\frac{1}{2}}
		-
		(v_3)_{i-\frac{1}{2},j+\frac{1}{2}}^{k+\frac{1}{2}} 
	}{\Delta x_1}
	\label{eqn:WWW_022}
	\\
	(\sigma_{32})^{k+1}_{i+\frac{1}{2},j} &=& 
	(\sigma_{32})^{k}_{i+\frac{1}{2},j} 
	+
	\mu \Delta t 
	\frac{
		(v_3)_{i+\frac{1}{2},j+\frac{1}{2}}^{k+\frac{1}{2}} 
		-
		(v_3)_{i+\frac{1}{2},j-\frac{1}{2}}^{k+\frac{1}{2}} 
	}{\Delta x_2}
	\label{eqn:WWW_023}
	\\
	(v_3)_{i+\frac{1}{2},j+\frac{1}{2}}^{k+\frac{1}{2}} 
	&=& 
	(v_3)_{i-\frac{1}{2},j+\frac{1}{2}}^{k-\frac{1}{2}} 
	+
	\frac{\rho}{\Delta t} \left\{
		\frac{(\sigma_{31})^k_{i+1,j+\frac{1}{2}} - (\sigma_{31})^k_{i,j+\frac{1}{2}} }{\Delta x_1}
		+
		\frac{(\sigma_{32})^k_{i+\frac{1}{2},j+1} - (\sigma_{32})^k_{i+\frac{1}{2},j} }{\Delta x_2}
	\right\} 
	\label{eqn:WWW_024}
\end{eqnarray}
これらの式を$k=0$から順次計算することで,速度と応力が,半時間ステップの間隔で交互に求められる.
なお,境界条件の与え方は,領域を矩形セルの集合としてどのように近似するかに依存する.
超音波探傷の数値シミュレーションでは,式(\ref{eqn:WWW_004})にもあるように,
トラクションが指定された条件を扱うことが多い.この場合,セルの境界を領域境界に
あわせるように領域を近似する.
その結果,計算領域の境界は階段状になるため,法線ベクトルは$\pm x_1$あるいは$\pm x_2$
方向に限定される.
従って,トラクションベクトルは応力成分に一致し,
\[
	\fat{n}=(\pm 1, 0) \rightarrow t_3^{(n)}=\pm \sigma_{31}, \ \ 
	\fat{n}=(0, \pm1 ) \rightarrow t_3^{(n)}=\pm \sigma_{32}
	\label{eqn:}
\]
となるので,境界上に位置する$(\sigma_{31})_{i,j+\frac{1}{2}},(\sigma_{32})_{i+\frac{1}{2},j}$
に,指定された境界値を代入すればよい.
$x_1$や$x_2$軸に対して傾いた境界と境界値の設定を厳密に行うことはできないが,
FDTD法ではこのような方法で与えられた境界条件を考慮して問題を数値的に解く.
\begin{figure}[h]
	\begin{center}
	\includegraphics[width=1.0\linewidth]{Figs/FDgrid_x.pdf} 
	\end{center}
	\caption{FDTD法における差分格子の空間配置.} 
	\label{fig:staggered}
\end{figure}
%%%%%%%%%%%%%%%%%%%%%%%%%%%%%%%%%%%%%%%%
\subsection{数値解析例}
以上に述べた方法で行った波動解析結果の一例を図\ref{fig:snapshot}に示す.
この図は,$M=32$素子からなるアレイ探触子の,13番目の素子($i=13$)から
送信した場合の波動場の進展を,3つの異なる時間におけるスナップショット
として示したものである.左側(a)-(c)は,領域内部に空洞を含むモデル,
右側(d)-(f)は割れを含むモデルでの計算結果である.
時刻$t$は送信素子を表現するトラクションの加振周期$T_0$で,
空間座標は波長$\lambda_0=c_TT_0$で,それぞれ無次元化されている.
アレイ素子の位置は黒の点で示した通りであり,送信兼受信素子となる
13番目のものだけが白の丸で描かれている.
アレイ素子は有限な幅をもち点源にはなっていないが,
ほぼ円筒波状の入射場が励起されており,素子端部での回折波の影響
はほとんど見られない.

素子幅は
波長よりも短く,素子端部での回折波
\begin{figure}[h]
	\begin{center}
	\includegraphics[width=1.0\linewidth]{Figs/snapshot.pdf} 
	\end{center}
	\caption{面外波動場のスナップショット(アレイ素子13から送信した場合の結果).} 
	\label{fig:snapshot}
\end{figure}
\begin{figure}[h]
	\begin{center}
	\includegraphics[width=0.8\linewidth]{Figs/Bscan_cavity.pdf} 
	\end{center}
	\caption{FDTD法で合成したアレイ観測波形.(a)走時波形, (b)Aスキャン波形. 
	空洞を含むモデルでアレイ素子13から送信した場合の結果.} 
	\label{fig:bscan_cavity}
\end{figure}
\begin{figure}[h]
	\begin{center}
	\includegraphics[width=0.8\linewidth]{Figs/Bscan_crack.pdf} 
	\end{center}
	\caption{FDTD法で合成したアレイ観測波形.(a)走時波形, (b)Aスキャン波形. 
	割れを含むモデルでてアレイ素子13から送信した場合の結果.} 
	\label{fig:bscan_crack}
\end{figure}
\begin{figure}[h]
	\begin{center}
	\includegraphics[width=1.0\linewidth]{Figs/cavity.pdf} 
	\end{center}
	\caption{画像化結果(円形空洞の場合).} 
	\label{fig:cavity}
\end{figure}
\begin{figure}[h]
	\begin{center}
	\includegraphics[width=1.0\linewidth]{Figs/crack.pdf} 
	\end{center}
	\caption{画像化結果(サイン波状の割れの場合).} 
	\label{fig:crack}
\end{figure}
