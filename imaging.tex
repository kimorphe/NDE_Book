%%%%%%%%%%%%%%%%%%%%%%%%%%%%%%%%%%%%%%%%%%%%%
\section{形状再構成問題の数値計算例}
\subsection{はじめに}
本節では、有限な大きさの散乱体を対象として,これまでに示した
方法に基づいて散乱体の形状再構成を行う.
散乱体は円形空洞と割れの2つをとりあげ,いくつかの異なる
画像化式によってイメージングを行い結果を比較する.
これらのイメージングには,数値シミュレーションによって
合成した散乱波の時間波形を用いる.
イメージングは開口合成法の一種とみなすことのできる
画像化関数を用い,全て時間領域における計算によって行う.
\subsection{問題設定}
図\ref{fig:imaging_problem}に,2次元イメージングを行う
2つのモデルを示す.この図の(a)は円形空洞を,(b)は滑らかに曲がった
割れを含む半無限領域を示している.ここでは,半無限領域の表面に設置した
リニアアレイ探触子から面外波(SH波)を送受信して散乱波形を取得し,
これら2つの欠陥について形状再構成(イメージング)を行う.
その際,送受信は全素子の組み合わせで行い,得られた波形の一部あるいは全部
を使ってイメージングを行う.
なお,図\ref{fig:imaging_problem}の寸法は,後にも述べる通り,
送信波の波長で無次元化したものである.また,図中の破線で示した範囲は画像化を行う領域を表す.
図\ref{fig:linear_array}はリニアアレイ探触子の素子配置を示す.この図では,アレイ探触子を構成する
素子数を$M$とし,半無限媒体の表面において,第$i$番目のアレイ素子が占める箇所を$e_i$として示している.
なお,各素子の幅$w$と,隣接する素子の間隔であるピッチ$p$は
一定としている.このとき,アレイ探触子の開口(aperture)$a_p$は
次式で与えられる.
\begin{equation}
	a_p=p(M-1)+w
	\label{eqn:aperture}
\end{equation}
図\ref{fig:imaging_problem}では,これらリニアアレイ探触子の配置を
\[
	p=1, w=0.5, M=32, (a_p=32)
\]
とし,アレイ探触子の中心が欠陥の直上に来るようにする.
図\ref{fig:imaging_problem}は,このようにして配置したアレイ素子の位置を,
媒体表面($x_2=30$)上の黒の点で表している.
\begin{figure}[h]
	\begin{center}
	\includegraphics[width=1.0\linewidth]{Figs/model.pdf} 
	\end{center}
	\caption{ 形状再構成問題のための数値計算モデル.}
	\label{fig:imging_problem}
\end{figure}
\begin{figure}[h]
	\begin{center}
	\includegraphics[width=0.6\linewidth]{Figs/array.pdf} 
	\end{center}
	\caption{リニアアレイセンサーを模擬した送受信領域の設定.} 
	\label{fig:linear_array}
\end{figure}
各アレイ素子は,送信時には予め指定された面外方向表面力$\bar{t}_3(t)$を,
素子範囲$e_j$内で一様に加えて面外波を励起する.
受信センサーとしては,面外方向の粒子速度を観測するものとする.
このとき,素子$i$で送信し,素子$j$で受信された波形$a_{ij}(t)$は,
素子$j$の範囲$e_j$で生じる速度場を次のように平均化したものとして与える.
\begin{equation}
	a_{ij}(t)=\frac{1}{w}\int_{e_j} v_3(\fat{x},t)ds, \ \ (i,j=1,2,\dots M)
	\label{eqn:WWW_001}
\end{equation}
ただし$v_3(\fat{x},t)$は,$(\fat{x},t)$における面外方向粒子速度を表す.
いま,送信素子の番号を$i$,送信開始時刻を$t=0$とし,それ以前には媒体は
静止した状態にあるとする.
半無限領域を$B$,その中で欠陥が占める領域を$D$とすれば,
面外波動場の支配方程式系は,以下のように表される.
\begin{equation}
	\rho \dot v_3 = \sigma_{31,1}+\sigma_{32,2}, 
	\ \ (\fat{x} \in B \setminus \bar{D},\, t>0)
	\label{eqn:WWW_002}
\end{equation}
\begin{equation}
	\sigma_{3\alpha}=\mu u_{3,\alpha}, 
	\ \ (\fat{x} \in B \setminus \bar{D},\, t>0)
	\label{eqn:WWW_003}
\end{equation}
\begin{equation}
	u_3(\fat{x},0)=0, v_3(\fat{x},0)=0, \ \ (\fat{x} \in B \setminus \bar{D})
	\label{eqn:WWW_003}
\end{equation}
\begin{equation}
	t_3^{(n)}(\fat{x},t)=
	\left\{
	\begin{array}{cc}
		0, & \left(\fat{x}\in (\partial H \setminus e_i ) \cup \partial D,\,  t>0 \right) \\
		\bar{t}_3(t), & \left(\fat{x}\in (\partial H\setminus e_i)  \cup \partial D,\, t>0 \right)\\
	\end{array}
	\right.
	\label{eqn:WWW_004}
\end{equation}
ただし,$\fat{x}=(x_1,x_2)$,$u_3(\fat{x},t)$は面外変位を,$\mu$はせん断剛性を
表し,$t^{(n)}_3$は,
\begin{equation}
	t^{(n)}_3= \sigma_{3\beta}n_\beta
	\label{eqn:WWW_004}
\end{equation}
で与えられる表面力ベクトルの面外方向成分である.なお,2次元問題のため,指標$\alpha,\beta$は
1または2をとる.
ここでは,送信波形を次のようにRicker波で与える.
\begin{equation}
	\bar{t}_3(t)=\left\{ 1-2\left(\pi f_0t\right)^2 \right\}\exp\left\{-(\pi f_0 t)^2\right\}
	\label{eqn:WWW_005}
\end{equation}
なお,$f_0$はピーク周波数を表し,式(\ref{eqn:WWW_005})のRicker波とその周波数スペクトルを
描くと図\ref{fig:Ricker}のようになる.
\begin{figure}[h]
	\begin{center}
	\includegraphics[width=0.8\linewidth]{Figs/ricker.pdf} 
	\end{center}
	\caption{入射波を励起するトラクション波形に用いる(a)Ricker波と
	(b)その周波数スペクトル.} 
	\label{fig:ricker}
\end{figure}
横波の位相速度$c_T$は,密度$\rho$とせん断剛性$\mu$を用いて,$c_T=\sqrt{\frac{\mu}{\rho}}$
で与えられ,周波数$f_0$における波長は
\begin{equation}
	\lambda_0 = c_T/f_0
	\label{eqn:WWW_006}
\end{equation}
で得られる.本節では,長さに関する量は波長$\lambda_0$で,時間に関する量は
周期$T_0=1/f_0$規格化して示す.このことは,図\ref{fig:imaging_problem}に
示した領域の大きさについても同様である.
%%%%%%%%%%%%%%%%%%%%%%%%%%%%%%%%%%%%%%%%
\subsection{散乱波形の合成 (FDTD法)}
イメージングに用いる波形$\left\{ a_{ij}(t)\left|i,j=1,2,\dots M\right.\right\}$は,
式(\ref{eqn:WWW_002})-(\ref{eqn:WWW_004})の初期値−境界値問題を
FDTD(finite-difference time-domain)法で解き,
式(\ref{eqn:WWW_001})に従って合成する.
FDTD法は有限差分法の一種で,構造格子上で方程式の離散化を行うため,
領域形状を正確に表現することはできない.その一方で,有限要素法や
境界要素法に比べて実装は簡単であり,計算効率も高い.
そのため,超音波探傷試験のシミュレーションのように,あまり高い
計算精度が要求されない問題には非常に使いやすい.
従って超音波イメージングの数値シミュレーションを目的とした
波形合成に有用であるため,以下,均質媒体における2次元面外波について,
FDTD法による離散化についての基本的な事柄を述べておく.

FDTD法では,式(\ref{eqn:WWW_002})と,式(\ref{eqn:WWW_003})の両辺を時間で微分して
得られる
\begin{equation}
	\dot \sigma_{3\alpha}=\mu v_{3,\alpha}
	\label{eqn:WWW_007}
\end{equation}
を連立した,1階の偏微分方程式系を$v_3,\sigma_{31}$および$\sigma_{32}$を
求めるべき未知の波動場として離散化する.
その際,時間に関してはリープフロッグ法を適用し,空間に関しては
スタガード格子を用い,いずれも中央差分で偏微分を近似する.

%\subsection{時間に関する離散化}
\subsection{時間微分の差分近似}
時間の関数$f(t)$を中央差分で離散化する.
そのために,時間ステップ長を$\Delta t$, $t=0$を基準に,第$k$番目の時間ステップ
を$t^k=k\Delta t$とする.$t=t^k$における$f(t)$の値を$f(t^{k})=f^k$とすれば,
半整数時間ステップ$t^{k+\frac{1}{2}}=(k+\frac{1}{2})\Delta t$における
$f(t)$の微分は$\dot f(t^{k+\frac{1}{2}})$は,中央差分近似により
\begin{equation}
	\dot{f}(t^{k+\frac{1}{2}})
	=\dot{f}^{k+\frac{1}{2}} \approx \frac{ f^{k+1}-f^k}{\Delta t}
	\label{eqn:WWW_008}
\end{equation}
とすることができる.
従って,式(\ref{eqn:WWW_008})を式(\ref{eqn:WWW_002})と式(\ref{eqn:WWW_007})に
あてはめると,
\begin{equation}
	\rho \frac{ (v_3)^{l+1}-(v_3)^l}{\Delta t}
	=
	(\sigma_{31,1})^{l+\frac{1}{2}}
	+
	(\sigma_{31,2})^{l+\frac{1}{2}}, 
	\ \ (k=0,\pm 1, \pm 2\dots )
	\label{eqn:WWW_009}
\end{equation}
と,
\begin{equation}
	\frac{ (\sigma_{3\alpha})^{k+1}-(\sigma_{3\alpha})^k}{\Delta t}
	=
	\mu (v_{3,\alpha})^{k+\frac{1}{2}},
	\ \ (k=0,\pm 1, \pm 2\dots )
	\label{eqn:WWW_010}
\end{equation}
を得る.ただしこれらの式では,近似であることを意味する$\simeq$ではなく
通常の等号($=$)を用いている.
式(\ref{eqn:WWW_009})と(\ref{eqn:WWW_010})を変形すると,
\begin{eqnarray}
	(v_3)^{l+1} &=& 
	(v_3)^l
	+
	\frac{\rho}{\Delta t} \left\{
	(\sigma_{31,1})^{l+\frac{1}{2}} + (\sigma_{31,2})^{l+\frac{1}{2}}
	\right\} 
	\label{eqn:WWW_011}\\
	(\sigma_{3\alpha})^{k+1} &=& 
	(\sigma_{3\alpha})^{k} 
	+
	\mu \Delta t (v_{3,\alpha})^{k+\frac{1}{2}}
	\label{eqn:WWW_012}
\end{eqnarray}
となるので,速度$v_3$に関する時間微分を近似するための時間ステップ$l$と,
応力$\sigma_{3\alpha}$のための時間ステップ$k$を
\begin{equation}
	l=k-\frac{1}{2} \Rightarrow t^k=t^l+\frac{\Delta t}{2}
	\label{eqn:WWW_013}
\end{equation}
と,互いに$\Delta t/2$だけずらして設定すれば,式(\ref{eqn:WWW_011})は
\begin{equation}
	(v_3)^{k+\frac{1}{2}} = (v_3)^{k-\frac{1}{2}}
	+
	\frac{\rho}{\Delta t} \left\{
	(\sigma_{31,1})^{k} + (\sigma_{31,2})^{k}
	\right\} 
	\label{eqn:WWW_014}
\end{equation}
となる.
従って,整数時間ステップ$t^k$での応力を用いて,半整数時間ステップ$t^{k+\frac{1}{2}}$
における速度が求められる.一方,式(\ref{eqn:WWW_012})では,
$t^{k+\frac{1}{2}}$における速度から$t^{k+1}$における応力が求められるので,
式(\ref{eqn:WWW_012})と式(\ref{eqn:WWW_014})を交互に用いることで,
応力と速度を順次初期値から更新して求められることが分かる.
このような時間ステッピングの方法をリープフロッグ法と呼ぶ.
\begin{figure}[h]
	\begin{center}
	\includegraphics[width=0.6\linewidth]{Figs/FDgrid_t.pdf} 
	\end{center}
	\caption{リープフロッグ法による時間に関する離散化.} 
	\label{fig:leapfrog}
\end{figure}
%%%%%%%%%%%%%%%%%%%%%%%%%5%
\subsection{空間微分の差分近似}
式(\ref{eqn:WWW_012})と式(\ref{eqn::WWW_014})の右辺には空間に関する微分
\[
	v_{3,1}=\frac{\partial v_3}{\partial x_1},\,
	v_{3,2}=\frac{\partial v_3}{\partial x_2}, \,
	\sigma_{31,1}=\frac{\partial \sigma_{31}}{\partial x_1},\,
	\sigma_{32,2}=\frac{\partial \sigma_{32}}{\partial x_2}
\]
が含まれる.これらの空間微分を差分近似するために,図\ref{fig:staggered}
にあるように,空間を幅$\Delta x_1$,高さ$\Delta x_2$のセルに区切り,
媒体が占める領域$V$を,これら小さなセルの集まりで近似する.
セルの頂点は
$(x_1)_i=i\Delta x_1$
$(x_2)_j=j\Delta x_2$


スタガード格子の$x_i$軸方向の間隔を
$\Delta x_i,(i=1,2)$とし,その間隔は一定とする.
この間隔で時空間内に配置された差分格子における関数$f(\fat{x},t)$の値を$f^k_{i,j}$と書くことにする.ただし,$k$は時間ステッ
プを,$i,\, j$は$x_1$および$x_2$方向の格子位置を表す番号を意味する.このとき,中央差分で時間微分を近似すれば,
\begin{equation}
        \dot{f}^{k+\frac{1}{2}} \approx \frac{ (f)^{k+1}-(f)^k}{\Delta t}
        \label{eqn:dfdt}
\end{equation}
となり,後に示すように,整数時間ステップ$k$で$f^k$を,半整数時間ステップ$t^{k+\frac{1}{2}}$において$\dot f$を,初期値$f^0, \dot f^{\frac{1}{2}}$から交互に求める蛙飛び差分(リープフロッグ)法による差分方程式が得られる.一方,空間微分$\nabla f$は
\begin{eqnarray}
        \left( \frac{\partial f}{\partial x_1}\right)
        _{i+\frac{1}{2}, \, j+\frac{1}{2}}
        & \approx &
                \frac{\left(f\right)_{i+1,j+\frac{1}{2}} -\left(f\right)_{i,\, j+\frac{1}{2}}}
                {\Delta x_1}
        \label{eqn:dfdx1}
        \\
        \left( \frac{\partial f}{\partial x_2}\right)
        _{i+\frac{1}{2}, \, j+\frac{1}{2}}
        & \approx &
        \frac{\left(f\right)_{i+\frac{1}{2},j+1} -\left(f\right)_{i+\frac{1}{2},\, j}}{\Delta x_2}
        \label{eqn:dfdx2}
\end{eqnarray}
と近似する.このような差分化を行うためには,関数$f$とその勾配$\nabla f$の計算格子がスタガード格子を形成するように配置して
おけばよい.

\begin{figure}[h]
	\begin{center}
	\includegraphics[width=1.0\linewidth]{Figs/FDgrid_x.pdf} 
	\end{center}
	\caption{FDTD法における差分格子の空間配置.} 
	\label{fig:staggered}
\end{figure}

%%%%%%%%%%%%%%%%%%%%%%%%%%%%%%%%%%%%%%%%
\subsection{画像化結果}
\begin{figure}[h]
	\begin{center}
	\includegraphics[width=1.0\linewidth]{Figs/cavity.pdf} 
	\end{center}
	\caption{画像化結果(円形空洞の場合).} 
	\label{fig:cavity}
\end{figure}
\begin{figure}[h]
	\begin{center}
	\includegraphics[width=1.0\linewidth]{Figs/crack.pdf} 
	\end{center}
	\caption{画像化結果(サイン波状の割れの場合).} 
	\label{fig:crack}
\end{figure}
