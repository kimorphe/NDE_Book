%%%%%%%%%%%%%%%%%%%%%%%%%%%%%%%%%%%%%%%%%%%%%
\section{形状再構成問題の数値計算例}
\subsection{はじめに}
本節では、有限な大きさの散乱体を対象として,これまでに示した
方法に基づいて散乱体の形状再構成を行う.
散乱体は円形空洞と割れの2つをとりあげ,いくつかの異なる
画像化式によってイメージングを行い結果を比較する.
これらのイメージングには,数値シミュレーションによって
合成した散乱波の時間波形を用いる.
イメージングは開口合成法の一種とみなすことのできる
画像化関数を用い,全て時間領域における計算によって行う.
\subsection{問題設定}
図\ref{fig:imaging_problem}に,2次元イメージングを行う
2つのモデルを示す.この図の(a)は円形空洞を,(b)は滑らかに曲がった
割れを含む半無限領域を示している.ここでは,半無限領域の表面に設置した
リニアアレイ探触子から面外波(SH波)を送受信して散乱波形を取得し,
これら2つの欠陥について形状再構成(イメージング)を行う.
その際,送受信は全素子の組み合わせで行い,得られた波形の一部あるいは全部
を使ってイメージングを行う.
なお,図\ref{fig:imaging_problem}の寸法は,後にも述べる通り,
送信波の波長で無次元化したものである.また,図中の破線で示した範囲は画像化を行う領域を表す.
図\ref{fig:linear_array}はリニアアレイ探触子の素子配置を示す.この図では,アレイ探触子を構成する
素子数を$M$とし,半無限媒体の表面において,第$i$番目のアレイ素子が占める箇所を$e_i$として示している.
なお,各素子の幅$w$と,隣接する素子の間隔であるピッチ$p$は
一定としている.このとき,アレイ探触子の開口(aperture)$a_p$は
次式で与えられる.
\begin{equation}
	a_p=p(M-1)+w
	\label{eqn:aperture}
\end{equation}
図\ref{fig:imaging_problem}では,これらリニアアレイ探触子の配置を
\[
	p=1, w=0.5, M=32, (a_p=32)
\]
とし,アレイ探触子の中心が欠陥の直上に来るようにする.
図\ref{fig:imaging_problem}は,このようにして配置したアレイ素子の位置を,
媒体表面($x_2=30$)上の黒の点で表している.
\begin{figure}[h]
	\begin{center}
	\includegraphics[width=1.0\linewidth]{Figs/model.pdf} 
	\end{center}
	\caption{ 形状再構成問題のための数値計算モデル.}
	\label{fig:imging_problem}
\end{figure}
\begin{figure}[h]
	\begin{center}
	\includegraphics[width=0.6\linewidth]{Figs/array.pdf} 
	\end{center}
	\caption{リニアアレイセンサーを模擬した送受信領域の設定.} 
	\label{fig:linear_array}
\end{figure}
各アレイ素子は,送信時には予め指定された面外方向表面力$\bar{t}_3(t)$を,
素子範囲$e_j$内で一様に加えて面外波を励起する.
受信センサーとしては,面外方向の粒子速度を観測するものとする.
このとき,素子$i$で送信し,素子$j$で受信された波形$a_{ij}(t)$は,
素子$j$の範囲$e_j$で生じる速度場を次のように平均化したものとして与える.
\begin{equation}
	a_{ij}(t)=\frac{1}{w}\int_{e_j} v_3(\fat{x},t)ds, \ \ (i,j=1,2,\dots M)
	\label{eqn:WWW_001}
\end{equation}
ただし$v_3(\fat{x},t)$は,$(\fat{x},t)$における面外方向粒子速度を表す.
面外波の励起開始時刻を$t=0$にとり,それ以前には媒体は静止した状態にある
とすれば,ここで考える面外波動場の支配方程式系は,以下のように表される.
\begin{equation}
	\rho \dot v_3 = \sigma_{31,1}+\sigma_{32,2}
	\label{eqn:}
\end{equation}
\begin{equation}
	\sigma_{3\alpha}=\mu u_{3,\alpha}
	\label{eqn:}
\end{equation}
\begin{equation}
	u_3(\fat{x},0)=0, v_3(\fat{x},0)=0
	\label{eqn:}
\end{equation}
\begin{equation}
	t_3^{(n)}(\fat{x},t)=0,
	\label{eqn:}
\end{equation}
ただし,$\fat{x}=(x_1,\, x_2)$,$u_3(\fat{x},t)$は面外変位を,
$t^{(n)}_3$は,
\begin{equation}
	t^{(n)}_3= \sigma_{3\beta}n_\beta
	\label{eqn:}
\end{equation}
で与えられる表面力ベクトルの面外方向成分である.
また,指標$\alpha,\beta$は1または2とする.
%%%%%%%%%%%%%%%%%%%%%%%%%%%%%%%%%%%%%%%%
\subsection{散乱波形の合成}
\begin{figure}[h]
	\begin{center}
	\includegraphics[width=0.6\linewidth]{Figs/FDgrid_t.pdf} 
	\end{center}
	\caption{リープフロッグ法による時間に関する離散化.} 
	\label{fig:leapfrog}
\end{figure}
\begin{figure}[h]
	\begin{center}
	\includegraphics[width=1.0\linewidth]{Figs/FDgrid_x.pdf} 
	\end{center}
	\caption{FDTD法における差分格子の空間配置.} 
	\label{fig:staggered}
\end{figure}

%%%%%%%%%%%%%%%%%%%%%%%%%%%%%%%%%%%%%%%%
\subsection{画像化結果}
\begin{figure}[h]
	\begin{center}
	\includegraphics[width=1.0\linewidth]{Figs/cavity.pdf} 
	\end{center}
	\caption{画像化結果(円形空洞の場合).} 
	\label{fig:cavity}
\end{figure}
\begin{figure}[h]
	\begin{center}
	\includegraphics[width=1.0\linewidth]{Figs/crack.pdf} 
	\end{center}
	\caption{画像化結果(サイン波状の割れの場合).} 
	\label{fig:crack}
\end{figure}
