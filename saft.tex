\documentclass[10pt,a4j,dvipdfmx]{jarticle}
%\usepackage{graphicx,wrapfig}
\usepackage{graphicx}
\usepackage{pdfpages}
%\usepackage{showkeys}
\setlength{\topmargin}{-1.5cm}
%\setlength{\textwidth}{16.5cm}
\setlength{\textheight}{25.2cm}
\newlength{\minitwocolumn}
\setlength{\minitwocolumn}{0.5\textwidth}
\addtolength{\minitwocolumn}{-\columnsep}
%\addtolength{\baselineskip}{-0.1\baselineskip}
%
\def\Mmaru#1{{\ooalign{\hfil#1\/\hfil\crcr
\raise.167ex\hbox{\mathhexbox 20D}}}}
%
\begin{document}
\newcommand{\fat}[1]{\mbox{\boldmath $#1$}}
\newcommand{\D}{\partial}
\newcommand{\w}{\omega}
\newcommand{\ga}{\alpha}
\newcommand{\gb}{\beta}
\newcommand{\gx}{\xi}
\newcommand{\gz}{\zeta}
\newcommand{\vhat}[1]{\hat{\fat{#1}}}
\newcommand{\spc}{\vspace{0.7\baselineskip}}
\newcommand{\halfspc}{\vspace{0.3\baselineskip}}
\bibliographystyle{unsrt}
%\pagestyle{empty}
\newcommand{\twofig}[2]
 {
   \begin{figure}
     \begin{minipage}[t]{\minitwocolumn}
         \begin{center}   #1
         \end{center}
     \end{minipage}
         \hspace{\columnsep}
     \begin{minipage}[t]{\minitwocolumn}
         \begin{center} #2
         \end{center}
     \end{minipage}
   \end{figure}
 }
%%%%%%%%%%%%%%%%%%%%%%%%%%%%%%%%%
%\vspace*{\baselineskip}
\begin{flushright}
	波動解析と非破壊評価\\
\end{flushright}
\begin{center}
	{\LARGE  \bf 超音波イメージング法の原理} \\
\end{center}
%%%%%%%%%%%%%%%%%%%%%%%%%%%%%%%%%%%%%%%%%%%%%%%%%%%%%%%%%%%%%%%%
\setcounter{section}{3}
\section{問題設定}
物体中に存在するき裂や空洞などの欠陥を、超音波エコー波形を用いて形状再構成する問題を考える。
以下、 図\ref{fig:arrayM}を参照し、簡単のため無限領域内部に存在する欠陥の形状再構成
について議論する.ここで、媒体は等方線形弾性体、超音波をスカラー波とし、
欠陥が占める領域を$D$、その境界を$\partial D$と表す。
スカラー波の送受信は、観測面$\cal M$内にとった$M$個の点:
\begin{equation}
	{\cal M} =\left\{ 
		\fat{y}_1, \fat{y}_2, \cdots \fat{y}_M
	\right\}
	\label{eqn:}
\end{equation}
のいずれかにある理想的な送受信子で行う.こでいう理想的とは、次のことを意味する。
\begin{itemize}
\item
送受信子の大きさは波長に比べて小さく点波源、あるいは、ポイントレシーバーとみなすことができる。
\item
送受信子の応答特性が既知、すなわち、指向性や周波数応答特性が知られている。
\item
送信子は励起する波動場を、受信子は観測する波動場を乱さない。すなわち、媒体中に発生する
波動場は送受信子と干渉せず、同一点で送信と受信を行うこともできる。
\end{itemize}
いま, 位置$\fat{y}_i$にある送信子から入射波を励起し、位置$\fat{y}_j$にある受信子で波動場を
観測したときに得られるAスキャン波形を$a_{ij}(t)$と表す。
このとき、全ての可能な送受信位置の組み合わせで得られ全波形データは
\begin{equation}
	{\cal D}_M= \left\{ a_{ij}(t) \left| (\fat{y}_i,\fat{y}_j)  \in {\cal M}\times {\cal M} \right.\right\}
	\label{eqn:full_mat}
\end{equation}
と表される。このような観測にアレイ探触子を用いる場合、送受信チャンネルを順次切り替えながら、全ての素子の
組み合わせで波形を収録する。このような観測方式はフルマトリクスキャプチャリングと呼ばれる。
この場合、$M$はアレイ素子の位置を、$\cal M$はアレイ探触子の物理的な開口を意味する。
同じ観測波形のセット${\cal D}_M$は、送信子一つと受信子一つを用い、それらの位置を変えて計測を
繰り返すことでも得られる。この場合、$M$は網羅される送受信点位置を、$\cal M$は一連の計測でカバーされる
開口(合成開口)を表す。波形収録の物理的な方法がどのようであっても、同じ波形データのセットが
得られる場合、画像化の結果も同じであることは明らかであろう。

同じ観測範囲$\cal M$において,一つの探触子で送受信を行うパルスエコー法で得られるデータは
\begin{equation}
	{\cal D}_0= \left\{ a_{ii}(t) \left| \fat{y}_i \in {\cal M}\right.\right\}
	\label{eqn:pulse_echo}
\end{equation}
と表わされる。なお、パルスエコーモードで観測された波形は、表記を簡単にするために
\begin{equation}
	a_i(t):=a_{ii}(t)
	\label{eqn:}
\end{equation}
と、2つめのインデックスを省略して書くことにする。このように多点で超音波計測を行う場合、
必ずしも全ての送受信位置の組で波形収録を行うとは限らない.従って、多点観測で得られる
波形データの集合は一般に${\cal D}_M$の部分集合となるので,これを$\cal D$と書くことにする.
${\cal D}_0$は$\cal D$の特別なケースである.以下では、波形データ$\cal D$や${\cal D}_0$が
与えられたときに、散乱体$D$の形状を再構成して画像として表示する方法について説明する。
その際、画像化を行う領域を$A$、画像化点(ピクセルあるいはボクセル)を$\fat{x}$とし、
画素値を$I(\fat{x})$と表す。画像関数$I(x)$は$A$におけるスカラー関数だが、その値が取りうる範囲は
画像化手法によって異なり正負いずれにもなり得る。また一部の方法では、理論的には
画素値が有界でない(無限大となる)ものもある。しかしながら、画像関数$I(\fat{x})$を数値的に評価する際、
数値誤差や不完全なデータのために有限な値に丸められることから、数値解析上$I(\fat{x})$の特異性
を気にする必要はない。
\begin{figure}[h]
	\begin{center}
	\includegraphics[width=0.6\linewidth]{Figs/measurement.eps} 
	\end{center}
	\caption{送受信アレイ$\cal M$,散乱体$D$および画像化領域$A$.} 
	\label{fig:arrayM}
\end{figure}
%
\section{開口合成法}
\subsection{再構成式}
\begin{figure}[h]
	\begin{center}
	\includegraphics[width=0.8\linewidth]{Figs/wvfm2.eps} 
	\end{center}
	\caption{2つの送受信子を使った点散乱体からのエコー観測.} 
	\label{fig:wvfm2}
\end{figure}
開口合成法では次のような波形の重ね合わせで欠陥の画像を合成する。
\begin{equation}
	I(x)=\sum_{\cal D}a_{ij}(t_{ij})w_iw_j
	\label{eqn:saft}
\end{equation}
ここに、$t_{ij}$は、送信点$\fat{y}_i$から入射された波が、画像化点$\fat{x}$を経て
観測点$\fat{y}_j$に至るまでに要する時間を表す.均質媒体では、伝播経路は
直線になると仮定するため、$t_{ij}$は位相速度$c$を用いて次のように表される.
\begin{equation}
	t_{ij}(\fat{x})=\frac{r_i(\fat{x})+r_j(\fat{x})}{c}, \ \ r_i:=\left| \fat{x} -\fat{y}_i\right|
	\label{eqn:tij}
\end{equation}
なお、和の記号$\sum_{\cal D}(\cdot)$は,データ$\cal D$についての和を取ることを意味する。
従って,${\cal D}= {\cal D}_M$ならばインデックス$i,j$に関する二重和を、${\cal D} ={\cal D}_0$ならば
$i(=j)$に関する和と解釈する.式(\ref{eqn:saft})の$w_i$と$w_j$は、送受信センサーの指向性や感度を補正
するための係数で、例えば波動場の距離による減衰を補償する場合には、位置$\fat{x}$に応じて与える。
この定式化は発見的なもので、式(\ref{eqn:saft})で再構成される画像$I(\fat{x})$のもつ意味を
理論的に解釈するためには別の画像化方法から式(\ref{eqn:saft})の形式が導出されることを見る必要がある.
一方,開口合成法がどのように機能するかを理解することは、次の節で見るように簡単で、
どのような計測条件が画像化のために有利であるかを容易に調べることができる。
また、再構成公式のアドホックな改善や拡張も容易で,例えば、送受信センサーの指向性や
距離減衰,入射波や散乱波の反射やモード変換を計測条件に応じて考慮することも(そのやり方が
妥当かどうかの検証はさておき)簡単である.
\subsection{画像合成原理}
開口合成法では、大きさをもった散乱体を互いに相互作用のない理想的な点散乱体の集まりとみなし、
これら点散乱体の位置と強度を推定する。点散乱体の強度は、点散乱体に入射された波とそこで発生する
散乱波強度の比を意味する。周波数領域で考えるとき、散乱強度は一般に複素数となる。
ただし、点散乱体には大きさが無いため、散乱体によるエネルギーの散逸を無視し、
以下では実数として扱う。なお、散乱体の物性値によっては入射波と散乱波の位相が反転するため、散乱強度は
負の値にも成り得る。これらのことから、画像化関数$I$は正負いずれの値も取り得る実数値関数で、
$I(\fat{x})$は位置$\fat{x}$にある点散乱体の推定された強度と解釈する。
従って、もしある位置$\fat{x}$で$I(\fat{x})=0$ならば、その位置には散乱体が存在しないことを意味する。
大きさをもった散乱体が、点散乱体のあつまりとしどの程度よく近似できるかという問題はさておき、
開口合成法が上のような意図をもったものである以上、少なくとも、単一の点散乱体の形状再構成は
条件が整えば満足できる精度で行うことができる必要がある。
また、単一点散乱体の開口合成結果を見ることで、開口合成法がどのように機能するかをよく理解することができる。
そこで、本節では、座標原点$\fat{o}$に存在する無指向性の単一点散乱体を、開口合成法によって画像化する
問題について調べておく。
なお、観測に用いる送受信子は、無指向性かつ周波数帯域が十分に広く、デルタ関数を含め
任意の波形を正確に送受信できると理想化された観測条件を仮定する。
\begin{figure}[h]
	\begin{center}
	\includegraphics[width=0.5\linewidth]{Figs/cloud.eps} 
	\end{center}
	\caption{点散乱体群による散乱体の表現(イメージ).} 
	\label{fig:cloud}
\end{figure}
\begin{figure}[h]
	\begin{center}
	\includegraphics[width=0.6\linewidth]{Figs/locii.eps} 
	\end{center}
	\caption{散乱波到達時刻における等時曲線.(a)送受信点が離れている場合,(b)送受信点が近接している場合.} 
	\label{fig:locii}
\end{figure}
\begin{figure}[h]
	\begin{center}
	\includegraphics[width=.6\linewidth]{Figs/saftM2.eps} 
	\end{center}
	\caption{開口合成画像の例.(a)送受信点が離れている場合,(b)送受信点が近接している場合.} 
	\label{fig:saftM2}
\end{figure}
\begin{figure}[h]
	\begin{center}
	\includegraphics[width=1.0\linewidth]{Figs/psf_pulse.eps} 
	\end{center}
	\caption{全周方向からパルス-エコーモードで送受信したときの開口合成像.(a)$M=10$, (b)$M=20$の場合.} 
	\label{fig:psf_pulse}
\end{figure}
\begin{figure}[h]
	\begin{center}
	\includegraphics[width=0.5\linewidth]{Figs/1D_prb.eps} 
	\end{center}
	\caption{反射壁の開口合成イメージング(1次元問題).} 
	\label{fig:1D_prb}
\end{figure}
\subsection{単一点散乱体の開口合成イメージング}
はじめに、一箇所で送受信を行った場合について考える。このとき、
\begin{equation}
	{\cal M} =\left\{ 
		\fat{y}_1
	\right\}
	\label{eqn:M1}
\end{equation}
で、波形データは
\begin{equation}
	{\cal D}_1=\left\{ a_{11}(t)\right\}
	\label{eqn:}
\end{equation}
である。従って、${\cal D}_1$から得られる画像は
\begin{equation}
	I(\fat{x})=a_{11}(t_{11})w_{11}
	\label{eqn:}
\end{equation}
である。ただし,$w_{11}=w_1w_1$で、この場合重み係数$w_{11}$は規格化係数としての意味しかない。
そこで、規格化された理想的なパルス波が受信されている状況を考え、
パルス幅$\Delta \tau$が観測時のサンプリング間隔$\Delta t$より小さいとする。
このとき、受信波中のパルス位置は、送信点$\fat{y}_1$と散乱体位置$\fat{o}$を
往復するために要する時間$t_{peak}$に等しく
\begin{equation}
	t_{peak}
	=
	\frac{2\left| \fat{y}_1 -\fat{o}\right|}{c}
	=
	\left. t_{11} \right|_{\fat{x}=\fat{o}}
	\label{eqn:t_peak}
\end{equation}
で与えられ、これ以外の時間では観測波形$a_{11}(t)$の振幅は0となる。
よって、開口合成画像は
\begin{equation}
	I(\fat{x})=a_{11}(t_{11}(\fat{x}))w_{11}=\left\{
	\begin{array}{cc}
		1 &  \left(t_{11}(\fat{x})=\frac{2|\fat{y}_1-\fat{o}|}{c}\right)\\
		0 &  ({\rm otherwise})
	\end{array}
	\right.
	\label{eqn:}
\end{equation}
となり、
\begin{equation}
	t_{11}(\fat{x})=\
	\frac{r_1(\fat{x})+r_1(\fat{x})}{c}
	=
	\frac{2|\fat{y}_1-\fat{o}|}{c}
	\label{eqn:}
\end{equation}
となる位置$\fat{x}$でのみ1,それ以外のところでは0となる画像を与える。
これは、$\fat{y}_1$を中心する半径$|\fat{y}_1-\fat{o}|$の円を表す。
このような画像からは、点散乱体が円弧上のいずれかの位置にあることが推定できる。
別の位置で送受信を行い、同様にして円弧を描けば、実際に点散乱体が存在する
位置で2つの円弧が交差するはずである。このことから、2次元問題では、
2点以上で観測を行うことで点散乱体の位置を特定することができる。

そこで次に2つの送受信位置をとり、$M=2$とした場合
\begin{equation}
	{\cal M} =\left\{ 
		\fat{y}_1, \fat{y}_2
	\right\}
	\label{eqn:M2}
\end{equation}
での画像化について考える。このとき、波形データは
\begin{equation}
	{\cal D}_2=\left\{ a_{11}(t), a_{12}(t), a_{21}(t), a_{22}(t)\right\}
	\label{eqn:}
\end{equation}
で,式(\ref{eqn:saft})は
\begin{equation}
	I(\fat{x})=a_{11}(t_{11})w_{11}+a_{12}(t_{12})w_{12}+a_{21}(t_{21})w_{21} + a_{22}(t_{22})w_{22}, \ \ (w_{ij}=w_iw_j)
	\label{eqn:I_M2}
\end{equation}
となる.波動場の相反性より$a_{12}(t)=a_{21}(t)$だから、いずれか一方を利用すればよく$w_{13}=0$としてよい。
また、距離減衰の効果を無視すれば$a_{11}(t)$と$a_{22}(t)$は、到達時間の差しかないので$w_{11}=w_{22}$を規格化係数にすることができる。
その結果、式(\ref{eqn:I_M2})は$w_{11}=1$として、
\begin{equation}
	I(\fat{x})=a_{11}(t_{11})+a_{12}(t_{12})+ a_{22}(t_{22})
	\label{eqn:I_M2d}
\end{equation}
と単純化できる。
右辺の項のうち$a_{12}(t)$は
\begin{equation}
	a_{12}(t)=\left\{
	\begin{array}{cc}
		1 &  \left( t=\frac{|\fat{y}_1-\fat{o}|+\left|\fat{y}_2-\fat{o}\right|}{c}\right)\\
		0 &  ({\rm otherwise})
	\end{array}
	\right.
	\label{eqn:}
\end{equation}
だから、$a_{12}(t_{21}(\fat{x}))$に相当する画像は、
$\fat{y}_1$と$\fat{y}_2$を焦点とするだ円になる。

開口合成法では,送信子$i$、受信子$j$で観測された波形$a_{ij}(t)$の時刻$t$に現れるエコーは、
位相速度を$c$として,
\begin{equation}
	\left| \fat{x} -\fat{y}_i\right|
	+
	\left| \fat{x} -\fat{y}_j\right|
	=ct
	\label{eqn:}
\end{equation}
となる位置$\fat{x}$に存在する散乱体から発生したもと考える.
ここで$\fat{x}$は画像化点(画素位置)を,$t_{ij}(\fat{x})$は画像化点の関数
であることに注意すると、$a_{11}(t_{11}(\fat{x}))$自体が一つの画像で,
$I(\fat{x})$は4つの画像の重ね合わせと言える.
いま,パルス状の入射波に対して点散乱体から発生したパルス状の散乱波が観測されているとする.
点散乱体の位置を原点にとると,観測波形$a_{11}(t)$は$t=r_i(\fat{o})/c=|\fat{y}_1|/c$近傍
にピークを持つ波形であると考えて良い.
パルス波の振幅を最大値で無次元化して1となるようにとれば,
$a_{11}(t_{11}())$は
\begin{equation}
	I_{11}(\fat{x}):=a_{11}(t_{11}(\fat{x}))=\left\{
	\begin{array}{cc}
		1 &  (ct_{11}(\fat{x})=|\fat{y}_1|))\\
		0 &  ({\rm otherwise})
	\end{array}
	\right.
	\label{eqn:}
\end{equation}
である. $ct_{11}(\fat{x})=|\fat{y}_1|))$は2次元問題では円の,3次元問題では球面の方程式を表す。
ここでは2次元画像を考えると、$I_11$は半径$|\fat{y}_1|$の円を表す画像で、円周上の点では画素値1,
それ以外の点では画素値0をもつ画像となる。つまり、この画像上では、点散乱体の位置が円周上のどこか
にあるということまでが特定できる。全く同様に考えると、$I_{22}$は中心$\fat{y}_2$で原点を通る円を
表す画像となっている。これら2つを重ね合わせると2つの円が重なった画像が得らる。 
この画像において2つの円が交わる点では画素値2,それ以外の円周上の点では画素値1,
他の場所では画素値0となる画像である。この画像が得られたとき、画素値2となる位置に点散乱体が
存在すると解釈することができる。さらに、
%%%%%
\subsection{散乱場の積分表現}
周波数領域におけるスカラー波の散乱問題を考える.
波動場を$u(\fat{x},\omega)$とし、特に必要の無い限り角周波数$\omega$は省略する。
$u(\fat{x})$は、ヘルムホルツ方程式を満足する.
\begin{equation}
	\nabla^2 u(\fat{x}) + k^2 u(\fat{x}) =0, \ \ (\fat{x} \notin \bar{D})
	\label{eqn:}
\end{equation}
ここに、$k=\omega/c$は一定の波数を表す。
はじめに,無限領域中に置かれた散乱体$D$は,境界$\partial D$上でディリクレ条件
\begin{equation}
	u(\fat{x}) = 0, \ \ \fat{x}\in \partial D
	\label{eqn:}
\end{equation}
あるいは,ノイマン条件:
\begin{equation}
	\frac{\partial u}{\partial n} = 0 ,\ \ \fat{x}\in \partial D
	\label{eqn:}
\end{equation}
を満足する場合について考える.
散乱場を$u^{\rm sc}(\fat{x})$,入射波$u^{in}(\fat{x})$とすると,全波動場は
\begin{equation}
	u(\fat{x})=u^{in}(\fat{x})+ u^{\rm sc}(\fat{x})
	\label{eqn:}
\end{equation}
と表される。散乱場は放射条件を満足するものとすれば、
$u^{\rm sc}(\fat{x})$は次のように積分表現することができる.
\begin{equation}
	u^{\rm sc}(\fat{y}) = \int _{\partial D} 
	\left\{
		G(\fat{x},\fat{y})t(\fat{x})
	-
		H(\fat{x},\fat{y})u(\fat{x})
	\right\} dS_x
	\label{eqn:}
\end{equation}
ここで、$G(\fat{x},\fat{y})$は
\begin{equation}
	\nabla^2 G(\fat{x},\fat{y}) + k^2 G(\fat{x},\fat{y}) =-\delta (\fat{x}-\fat{y} )
	\label{eqn:}
\end{equation}
を満足する無限領域のおけるグリーン関数で、
\begin{equation}
	G(\fat{x},\fat{y}) =\frac{e^{ikr}}{4\pi r}, \ \ r=\left| \fat{x}-\fat{y} \right|
	\label{eqn:Green}
\end{equation}
で与えられる.また,
\begin{equation}
	H(\fat{x},\fat{y}) = \frac{\partial G(\fat{x},\fat{y})}{\partial n(\fat{x})}, \ \ 
	t(\fat{x}) =\frac{\partial u(\fat{x})}{\partial n(\fat{x})}
	\label{eqn:}
\end{equation}
とした。
ここで、$y=\left|\fat{y}\right|\gg \left| \fat{x} \right|$のとき、幾何光学近似
\begin{equation}
	r\sim  y  -\hat{\fat{y}}\cdot \fat{x}, \ \ (y\rightarrow \infty)
	\label{eqn:para}
\end{equation}
を用いると、散乱場の積分表現は、ディリクレ問題,ノイマン問題、それぞれの場合について
\begin{equation}
	u^{\rm sc}(\fat{y}) \sim \frac{e^{iky}}{4\pi y} U(\hat{\fat{y}};s)  
	\label{eqn:}
\end{equation}
\begin{equation}
	U(\hat{\fat{y}};s)  = \int _{\partial D} t(\fat{x};s)e^{-ik\hat{\fat{y}}\cdot \fat{x}} dS_x
	, \ \ ( {\rm Dirichlet})
	\label{eqn:}
\end{equation}
\begin{equation}
	U(\hat{\fat{y}};s)  =-\int _{\partial D} u(\fat{x};s)e^{-ik\hat{\fat{y}}\cdot \fat{x}} dS_x
	, \ \ ({\rm Neumann})
	\label{eqn:}
\end{equation}
なお、$;$に続く$s$は入射波を指定するパラメータを表し,このように表記すること
入射波に依存した量であることを必要あ場合には明示する.
ここで、任意の$D$と場の量$f(\fat{x})$に対して次のように作用する特異関数$\gamma_D(\fat{x})$
\begin{equation}
	\int \gamma_D(\fat{x})f(\fat{x})d^3\fat{x} =\int _{\partial D} f(\fat{x}) dS
	\label{eqn:}
\end{equation}
を導入する.これを用いると、散乱振幅を
\begin{equation}
	U(\hat{\fat{y}};s)  = \int \gamma_D(\fat{x})t(\fat{x};s)e^{-ik\hat{\fat{y}}\cdot \fat{x}} dS_x
	, \ \ ( {\rm Dirichlet})
	\label{eqn:}
\end{equation}
\begin{equation}
	U(\hat{\fat{y}};s)  =-\int \gamma_D(\fat{x})u(\fat{x};s)e^{-ik\hat{\fat{y}}\cdot \fat{x}} dS_x
	, \ \ ({\rm Neumann})
	\label{eqn:}
\end{equation}
と、フーリエ積分の形式で表すことができる。

続いて,領域$D$を介在物が占める場合について考える.
ここで、介在物の外部で波数は$k_0$で一定とし、内部では一般に$k_0$とは異なる
波数となるとする。
そこで、波数を位置の関数として$k(\fat{x})$とすれば,支配方程式であるHelmholtz方程式は
\begin{equation}
	\nabla^2 u(\fat{x},\omega) + k^2(\fat{x}) u(\fat{x},\omega) =0
	\label{eqn:Hlmhlz2}
\end{equation}
となる。ここで、
\begin{equation}
	n_r(\fat{x}) =\frac{k}{k_0}
	\label{eqn:}
\end{equation}
\begin{equation}
	V(\fat{x})=n_r^2(\fat{x}-1)
	\label{eqn:}
\end{equation}
とおけば、
\begin{equation}
	k^2=k_0^2 + k_0^2 V(\fat{x})
	\label{eqn:}
\end{equation}
より、式(\ref{eqn:Hlmhlz2})は
\begin{equation}
	\nabla^2 u(\fat{x},\omega) + k_0(\fat{x}) u(\fat{x},\omega) = -k_0^2 V(\fat{x})u(\fat{x})
	\label{eqn:Hlmhlz3}
\end{equation}
と、形式上、非斉次のHelmholtz方程式の形に書き直すことができる。
従って、右辺の非斉次項を物体力項として扱えば、散乱場を次のように積分表示することができる。
\begin{equation}
	u^{\rm sc}(\fat{y})=u(\fat{y})-u^{in}(\fat{y}) = k_0^2\int G_0(\fat{x},\fat{y})V(\fat{x})u(\fat{x}) d^3\fat{x}
	\label{eqn:LS}
\end{equation}
ここで$G_0(\fat{x},\fat{y})$は波数$k_0$をもつ均質な無限媒体に対するグリーン関数で、
式(\ref{eqn:Green})において$k=k_0$としたものである.
式(\ref{eqn:LS})はLippman-Schwinger方程式、$V(\fat{x})$は散乱ポテンシャルと呼ばれる。
散乱ポテンシャルは$D$を台に持つため、$V(\fat{x})$を求めることができれば、
散乱体形状が再構成される。
最後に、$G_0(\fat{x},\fat{y})$に幾何光学近似(\ref{eqn:para})を用いれば、
式(\ref{eqn:LS})は
\begin{equation}
	u^{\rm sc}(\fat{y})
	=
	\frac{e^{ik_0y}}{4\pi y}
	\int
	k_0^2 V(\fat{x})u(\fat{x}) e^{-ik\hat{\fat{y}}\cdot \fat{x}}d^3\fat{x}
	\label{eqn:}
\end{equation}
となるので、
\begin{equation}
	U(\hat{\fat{y}},s)
	=
	\int
	k_0^2 V(\fat{x})u(\fat{x};s) e^{-ik\hat{\fat{y}}\cdot \fat{x}}d^3\fat{x}
	\label{eqn:U_vol}
\end{equation}
とすることで、散乱場を表面散乱体の場合と同じ形式に表すことができる。
%
\subsection{線形化逆散乱解析}
はじめに、式(\ref{eqn:LS})に基づき,散乱ポテンシャル$V(\fat{x}$を推定することで
介在物の形状を再構成する方法を示す.いま,散乱体$D$に平面波:
\begin{equation}
	u^{in}(\fat{x})= e^{ik_0\fat{p}\cdot\fat{x}}
	\label{eqn:}
\end{equation}
を入射したときの$U(\hat{\fat{y}};s)$を
\begin{equation}
	u^{\infty}(\hat{\fat{y}};\fat{p})=U(\hat{\fat{y}};\fat{p})
	\label{eqn:}
\end{equation}
と表し、これを散乱振幅と呼ぶ.
ここで、$\fat{p}$は入射平面波の伝播方向を表す単位ベクトルを意味する。
このとき、式(\ref{eqn:U_vol})にボルン近似を用い、$u(\fat{x})\simeq u^{in}(\fat{x})$
とすれば,散乱振幅は
\begin{equation}
	u^{\infty}(\hat{\fat{y}},\fat{p})
	=
	\int
	k_0^2 V(\fat{x}) e^{-ik(\hat{\fat{y}}-\fat{p} )\cdot \fat{x}}d^3\fat{x}
	\label{eqn:U_Born}
\end{equation}
と書ける。そこで,散乱ポテンシャル$V(\fat{x})$のフーリエ変換を
\begin{equation}
	\tilde V(\fat{k}) = \int V(\fat{x}) e^{i\fat{k}\cdot \fat{x}} d^3\fat{x}
	\label{eqn:Vk}
\end{equation}
とすれば、式(\ref{eqn:U_Born})は
\begin{equation}
	u^{\infty}(\hat{\fat{y}},\fat{p}) =k_0^2 \tilde V (
	k(\hat{\fat{y}}-\fat{p} )
	)
	\label{eqn:}
\end{equation}
となることから、散乱振幅が散乱ポテンシャルの波数ベクトル$k(\hat{\fat{y}}-\fat{x})$
でのスペクトル成分であることが分かる.
従って、送受信条件や周波数帯域を適切に設定し,十分な範囲と密度で波数スペクトル成分を
得ることができれば,逆フーリエ変換
\begin{equation}
	V(\fat{x}) =\frac{1}{(2\pi)^3} \int k_0^{-2} u^{\infty}(\hat{\fat{y}};\fat{p}) 
	e^{i\fat{k}\cdot \fat{x}}d^3\fat{k}
	\label{eqn:Vx}
\end{equation}
によって散乱ポテンシャル$V(\fat{x})$を再構成できる。
ここに,式(\ref{eqn:Vx})における波数ベクトル$\fat{k}$は
\begin{equation}
	\fat{k}=k(\hat{\fat{y}}-\fat{p})
	\label{eqn:}
\end{equation}
だから,フーリエ積分の評価に用いることのできる波数ベクトル成分は,角周波数$\omega$,
観測方向$\hat{\fat{y}}$,入射方向$\fat{p}$に依って決まる.
いま、$\omega$と$\fat{p}$を固定し、観測方向だけを自由に取ることができる場合を考えると、
$\fat{k}$は波数空間で、図\ref{fig:Evald}に実線で示したような半径$k$球を描き、
これをEvald球と呼ぶ。つまり、観測方向だけを変化させて観測した場合、
散乱ポテンシャルのEvald球状の成分だけしか得られない。一方、観測方向$\hat{\fat{y}}$に加え、
入射方向$\fat{p}$も任意に選べる場合、Evald球は原点を中心として半径$2k$の球表面と内部を履く.
このような波数空間の球領域はEvald limitting sphere(Evald限界球)と呼ばれる.
全ての送受信方向で観測を行うことができれば、Evald限界球内部の
波数成分全ての情報が得られるため、波数$k$十分に大きければ、単一の波数成分
のデータだけで、散乱ポテンシャルの十分な情報が得られる。
しかしながら、例えば超音波探傷試験では、送受信方向は制約が厳しく、
ごく限られた方向からの送受信しか行うことができないために、Evald限界球内の
一部のデータしか得られない。
一方で、送受信することのできる周波数帯域はパルスやチャープ波を用いることで
ある程度広くとることができる。このときには、波数$k$が$k\in \left(k_{min},k_{max} \right)$
の範囲にあるEvald球の一部が掃く領域が単一周波数(monochromatic)での
検査に比べて広くとることができ,散乱ポテンシャルに関するより多くの情報を得ることができる。
\\

ここで,散乱振幅を
\begin{equation}
	u^{\infty}(\hat{\fat{y}};\fat{p})=4\pi ye^{-iky} u^{sc}(\fat{y})
	\label{eqn:}
\end{equation}
として、観測波形データ$u^{sc}$を用いて式(\ref{eqn:Vx})を書き直すと
\begin{equation}
	V(\fat{x}) =\frac{1}{2\pi^2} \int k_0^{-2} yu^{sc}(\fat{y})
	e^{i\fat{k}\cdot \fat{x}-iky}d^3\fat{k}, 
	\ \ (\fat{k}=k(\hat{\fat{y}}-\fat{p}))
	\label{eqn:Vx_usc}
\end{equation}
となる。この再構成式がどのような操作を意味するかは、次節で述べるように、
周波数$\omega$に関する積分を行い時間領域における再構成式に直すことで
明らかになる。
\begin{figure}[h]
	\begin{center}
	\includegraphics[width=0.5\linewidth]{Figs/setup.eps} 
	\end{center}
	\caption{無限領域におけるスカラー波の散乱問題.} 
	\label{fig:}
\end{figure}
\begin{figure}[h]
	\begin{center}
	\includegraphics[width=0.6\linewidth]{Figs/Evald.eps} 
	\end{center}
	\caption{波数ベクトル空間におけるEvaldおよびEvald limitting sphere.} 
	\label{fig:Evald}
\end{figure}
\subsection{時間領域における再構成式}
式(\ref{eqn:Vx})の積分を評価するにあたり,微小体積要素$d^3\fat{k}$を 
$\fat{p}$,$\hat{\fat{y}}$, $k$を用いて表す必要がある.
%波数ベクトル$\fat{k}$は入射波の送信方向$\fat{p}$,散乱波の観測方向$\hat{\fat{y}}$と
%角周波数$\omega$に依存する.従って,波数ベクトルに関するフーリエ変換を観測波形を
%使って評価する際には、観測条件に応じた積分変数を取る必要がある。
ここでは,次のような二つのケースについて考える.
%では微小体積要素$d^3\fat{k}$を書き下すことができる。
\begin{enumerate}
\item
	送信(受信)方向$\fat{p}(\hat{\fat{y}})$を固定し、
	受信(送信)方向$\hat{\fat{y}}(\fat{p})$を変化させる場合\\

	観測方向を表すベクトルを次のように球座標を使って表す.
	\begin{equation}
		\hat{\fat{y}}=(\sin\theta\cos\phi,\sin\theta\sin\phi, \cos\theta)
		\label{eqn:}
	\end{equation}
	このとき,微小体積要素はヤコビ行列式を計算すれば
	\begin{equation}
		d^3\fat{k}=k^2\sin\theta (1-\hat{\fat{y}}\cdot\fat{p})dkd\theta d\phi
		\label{eqn:}
	\end{equation}
	となることが示される.
	ここで,
	\begin{equation}
		\cos\delta=\hat{\fat{y}}\cdot\fat{p}
		\label{eqn:}
	\end{equation}
	とおけば,
	\begin{equation}
		d^3\fat{k}=4k^2\sin\theta \sin^2 \frac{\delta}{2} dkd\theta d\phi
		\label{eqn:}
	\end{equation}
\item
	送信方向と受信方向が成す角度を一定に保ち、送受方向を同時に変化させる場合\\

	単位ベクトル:
	\begin{equation}
		\hat{\fat{k}}=\frac{\hat{\fat{y}}-\fat{p}}
		{\left| \fat{y}-\fat{p} \right| }
		\label{eqn:}
	\end{equation}
	を用いれば,波数ベクトル$\fat{k}$は
	\begin{equation}
		\fat{k}= k\left| \hat{\fat{y}}-\fat{p} \right| \hat{\fat{k}} 
		\label{eqn:}
	\end{equation}
	と書くことができ,$\hat{\fat{y}} \cdot \fat{p}= \cos \delta$とおけば,	
	\begin{equation}
		\fat{k}= 2k \sin \frac{\delta}{2}\hat{\fat{k}} 
	\label{eqn:}
	\end{equation}
	と表すことができる.$\delta$は一定だから,$\hat{\fat{k}}$を
	\begin{equation}
		\hat{\fat{k}}=(\sin\theta\cos\phi,\sin\theta\sin\phi, \cos\theta)
		\label{eqn:}
	\end{equation}
	と球座標を使って表せば,微小体積要素を
	\begin{equation}
		d^3\fat{k}=2\sin\frac{\delta}{2} 
		k^2 \sin\theta dkd\theta d\phi
		\label{eqn:}
	\end{equation}
\end{enumerate}
\begin{figure}[h]
	\begin{center}
	\includegraphics[width=0.7\linewidth]{Figs/pulse_echo.eps} 
	\end{center}
	\caption{送受信パターン.} 
	\label{fig:}
\end{figure}
パルスエコー法(一探触子法)による計測は後者の特別な場合に相当する。
以上より、$d^3\fat{k}$は
\begin{equation}
	d^3\fat{k}=k^2\hat{J}(\theta,\phi) dkd\theta d\phi
	\label{eqn:}
\end{equation}
と表すことができる.
これを式(\ref{eqn:Vx_usc})に代入することにより次の再構成式が得られる.
\begin{equation}
	V(\fat{x}) =\frac{1}{2\pi^2} \int yu^{sc}(\fat{y})
	e^{i\fat{k}\cdot \fat{x}-iky}
	\hat{J}(\theta,\phi) dkd\theta d\phi
	\label{eqn:}
\end{equation}

ここで,送信点を$\fat{z}$とし,
\begin{equation}
	u^{in}(\fat{x})= F(\omega) e^{ik_0\fat{p}\cdot(\fat{x}-\fat{z})}
	\label{eqn:}
\end{equation}
のような周波数成分$F(\omega)$をもつ平面波を入射する場合について考える。
このとき、散乱場は
\begin{equation}
	u^{\rm sc}(\fat{y})= \frac{e^{ik_0(y-\fat{p}\cdot \fat{z})}}{4\pi y }F(\omega)U(\hat{\fat{y}};\fat{p}) 
	\label{eqn:}
\end{equation}
と書くことができる.よって、散乱ポテンシャルは観測データとして得られる
散乱場を使って次のように表される.
\begin{equation}
	V(\fat{x}) =\frac{1}{2\pi^2} \int \frac{u^{sc}(\fat{y})}{F(\omega)}
	e^{i\fat{k}\cdot \fat{x}-iky+ik\fat{p}\cdot\fat{z}}
	\hat{J} dkd\phi d\theta	
	\label{eqn:}
\end{equation}
さらに,$\fat{k}=k(\hat{\fat{y}}-\fat{p})$を代入し,$k=\omega/c$を用いれば,
\begin{equation}
	V(\fat{x}) =\frac{1}{2\pi^2c} \int \frac{yu^{sc}(\fat{y})}{F(\omega)}
	e^{-ik_0(y-\hat{\fat{y}}\cdot\fat{x}+\fat{p}\cdot(\fat{x}-\fat{z})) }
	\hat{J} d\omega d\phi d\theta	
	\label{eqn:}
\end{equation}
を得る.
ここで、角周波数$\omega$に関するフーリエ変換
\begin{equation}
	a(\fat{y},\fat{z},t):=\frac{1}{2\pi} \int \frac{u^{sc}(\fat{y})}{F(\omega)}e^{-i\omega t}d\omega
	\label{eqn:}
\end{equation}
で得られる時間波形を用いると、散乱ポテンシャルを
\begin{equation}
	V(\fat{x}) =\frac{1}{2\pi^2c} \int a\left(\fat{y}, \fat{z}, t_{in}+t_{sc} \right) y \hat{J} d\theta d\phi
	\label{eqn:Vx_time}
\end{equation}
と表される。ただし、$t_{in},t_{sc}$はそれぞれ
\begin{equation}
	t_{in}=\frac{\fat{p}\cdot \left( \fat{x}-\fat{z}\right)}{c}, \ \ 
	t_{sc}=
	\frac{ y-\hat{\fat{y}}\cdot \fat{x} }{c} \simeq 
	\frac{\left|\fat{y}-\fat{x}\right|}{c} 
	\label{eqn:}
\end{equation}
で、入射波と散乱波の伝播時間を意味する。
式(\ref{eqn:Vx_time})を離散的な$\fat{y},\fat{z}$について評価する場合,
積分は次のような和とみなすことができ、これは、発見的に導かれた開口合成法の
形式と一致する.
\begin{equation}
	V(\fat{x}) = 
	\frac{1}{2\pi^2c} 
	\sum_{\fat{y},\fat{z}} 
	a\left(\fat{y}, \fat{z}, t_{in}+t_{sc} \right) y \hat{J} \Delta \theta \Delta \phi
	\label{eqn:}
\end{equation}
\end{document}
