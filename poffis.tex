%%%%%%%%%%%%%%%%%%%%%%%%%%%%%%%%%%%%%%%%%%%%%
\section{線形化逆散乱解析}
線形化逆散乱解析では,観測波形を用いて散乱体形状を表す特異関数を再構成する.
そのために,散乱波動場を特異関数を含む積分で表現し,観測波形と散乱体形状
を結びつける式を導く.散乱波の積分表現には,特異関数と散乱場が未知量として
含まれる.そこで,高周波近似や弱散乱体近似を使って散乱体表面や内部の場を表現し,
特異関数だけを未知量とする表現式に書き直す.その結果をフーリエ積分の形に帰着さ
せ,反復法を用いることなくフーリエ逆変換によって特異関数について解く.
本節では,これらの手順を順を追って説明するとともに,線形化逆散乱解析による
特異関数の再構成式が,時間領域では開口合成法と同じ形式をもつことを示す.
以下では,散乱波の積分表現を,表面散乱体と介在物の場合について示す.
次に,これらの散乱体形状を表現する特異関数を導入し,
特異関数を使った散乱場の積分表現に書き直す.
散乱場の積分表現は,散乱体タイプによらず同様な形式をもつため,ここでは
介在物の場合を取り上げ,形状再構成のための式を導く.
そのために,媒体中の散乱波を弱散乱体近似であるボルン近似を使って既知の場
として表現し,特異関数に関する方程式を得る.このようにして得られた方程式は,
波数ベクトルに関するフーリエ積分となっていることが示され,
そのフーリエ逆変換を取ることで,特異関数を再構成するための式を得る.
最後に,再構成式に含まれる周波数に関する積分を実行すれば,時間領域での
再構成式が得られ,その結果は開口合成法と同じ形式となることが示される.
なお,以上のことを,ここでは3次元スカラー波について行うが,
2次元問題や動弾性問題へもほとんど同じ手順で再構成式を導くことができる.
その詳細については,既存の書籍等に示されている(参考文献***).
\\
最終的に得られる再構成式,すなわち,画像化関数が開口合成法と
線形化逆散乱解析法で同様なものになるにせよ,線形化逆散乱解析の
としての定式化を行うことには次のような意義がある.

第一に,線形化逆散乱解析では,再構成する対象が明確に定義されている.
これに対して開口合成法では,画像化関数が正確に何を表現しているのか
は明確でなく,理想的な計測条件でのPSFを除き,画像化関数の陽な表現
も得られない.従って,画像化性能を理論的に調べることは難しい.
この理由から,本章では,代表的な条件での画像化結果を見ることで,
開口合成法の像合成の原理についてできるだけ一貫した考察を与える
ことを行ってきた.
これに対して線形化逆散乱解析では,観測波形と目的関数である特異関数の
関係がフーリエ変換で与えられる.従って,観測波形が像合成にどのように
貢献するかは、フーリエ解析の理論によって調べることができ,
理想的な観測条件について知ることや,観測条件の最適化が可能となる.
さらに,画像化アルゴリズムの実装では,離散フーリエ級数の理論に基づき,
高速化や適切な離散化方針を検討することができる.

開口合成法と線形化逆散乱解析が同様な画像化関数を与えるのであれば,
線形化逆散乱解析法に当てはまることは,概ね開口合成についても同様
と考えてよい.
例えば,線形化逆散乱解析の適用において適切な観測点の配置は,
開口合成法に対しても適切なものであるということが言える.
この意味で,線形化逆散乱解析について知ることは,開口合成法
の像合成原理と挙動について間接的に理論解析を行っているとみなすことができる.


線形化逆散乱解析のもう一つの利点に,計測モデルを画像化に
組み入れることが比較的容易であることが挙げられる.
線形化逆散乱解析は,散乱波動場の積分表現を基礎にもつ.
従って,どのような波動場を対象として適用すべきか明確である.
通常,観測で得られる量と,線形化逆散乱解析が対象とする
波動場の量は同じでない.そのため,観測量を波動場の量に変換する
モデル(観測モデル)が必要となる.
観測モデルを波動場の量について解くことができれば,
線形化逆散乱解析の定式化に観測モデルを組み込み,観測量を使って
画像化関数(再構成式)を書き下すことは簡単である.
超音波探傷システムのモデル化に通常用いられる線形システム論
では,このような取扱は特に周波数領域において非常に簡単である.
勿論,このようなモデルベースの画像化関数を,開口合成法の形式
に書くことも可能であり,線形化逆散乱解析法の修正を
開口合成法に持ち込むことも可能である.

開口合成法は,厳密な定式化に基づくものでない反面,様々な波動現象に関する
イメージングに適用でき,実装も非常に簡単である.このように極めて柔軟な方法
である開口合成法の適用範囲や最適化が, 線形化逆散乱解析との関係のもとで
可能となることが,2つの手法関連を整理することの意義と言える.
同じことは,時間反転法と開口合成法の関係についても言える.
時間反転法は,随伴方程式法やトポロジー勾配法といった最適化
理論として厳密な定式化と意味付けができる.
一方で,どのような観測条件が有利となるかといった点には,
最適化理論自体は答えてくれない.これに対し,
時間反転法は開口合成法と線形化逆散乱解析法の関係が分かっている
ため,結論としては,線形化逆散乱法に適した観測点の配置は
時間反転法や随伴方程式法でも(少なくとも,反復を行わない
フルウェーブインバージョンでは)同様と考えられることは明らかであろう.
%POFFIS<--時間域、送受パターンを制限--> SAFT <---sampling function 一般化/特殊化---> time-refersal 
%<--- 最適化理論---> FWI, Topological gradient method
\subsection{散乱場の積分表現}
無限領域中の散乱体$D$によるスカラー場$u(\fat{x},\omega)$の散乱問題周波数領域で考える.
以下,波動場が散乱体内部に浸透できない表面散乱体の場合,浸透可能な体積散乱体(介在物)の場合について,
順に散乱場の積分表現を示す.なお,必要の無い限り,角周波数$\omega$は引数から省略する.
\subsubsection{表面散乱体}
波動場$u(\fat{x})$は,次のヘルムホルツ方程式を満足する.
\begin{equation}
	\nabla^2 u(\fat{x}) + k^2 u(\fat{x}) =0, \ \ (\fat{x} \notin \bar{D})
	\label{eqn:ZZZ_00}
\end{equation}
ここで,$k=\omega/c$は波数を表し,位相速度$c$は一定とする.
無限領域中に置かれた散乱体$D$は,境界$\partial D$でディリクレ条件: 
\begin{equation}
	u(\fat{x}) = 0, \ \ \fat{x}\in \partial D
	\label{eqn:ZZZ_01}
\end{equation}
あるいは,ノイマン条件:
\begin{equation}
	\frac{\partial u(\fat{x})}{\partial n_x} = 0 ,\ \ \fat{x}\in \partial D
	\label{eqn:ZZZ_02}
\end{equation}
を満たすとする.ここで,$\frac{\partial }{\partial n_x}$は,位置$\fat{x}$における
外向き法線$\fat{n}(\fat{x})$方向の微分を表す.散乱場を$u^{\rm sc}(\fat{x})$,
入射波$u^{in}(\fat{x})$とすると,全波動場は
\begin{equation}
	u(\fat{x})=u^{in}(\fat{x})+ u^{\rm sc}(\fat{x})
	\label{eqn:ZZZ_03}
\end{equation}
と表される.入射場は散乱体が存在しないときの波動場を,全波動場は
散乱体が存在する場合の波動場を意味し,散乱場は放射条件を満足するものと考える.
このとき,$u^{\rm sc}(\fat{x})$は次のように積分表現することができる.
\begin{equation}
	u^{\rm sc}(\fat{y}) = \int _{\partial D} 
	\left\{
		G(\fat{x},\fat{y})t(\fat{x})
	-
		H(\fat{x},\fat{y})u(\fat{x})
	\right\} dS_x
	\label{eqn:ZZZ_04}
\end{equation}
ここで,$G(\fat{x},\fat{y})$は
\begin{equation}
	\nabla_x^2 G(\fat{x},\fat{y}) + k^2 G(\fat{x},\fat{y}) =-\delta (\fat{x}-\fat{y} )
	\label{eqn:ZZZ_05}
\end{equation}
を満足する全空間のグリーン関数で,
\begin{equation}
	G(\fat{x},\fat{y}) =\frac{e^{ikr}}{4\pi r}, \ \ r=\left| \fat{x}-\fat{y} \right|
	\label{eqn:ZZZ_Green}
\end{equation}
で与えられる.また,$H(\fat{x},\fat{y})$と$t(\fat{x})$は,次のような$G(\fat{x},\fat{y})$の
法線微分である.
\begin{equation}
	H(\fat{x},\fat{y}) = \frac{\partial G(\fat{x},\fat{y})}{\partial n_x}, \ \ 
	t(\fat{x}) =\frac{\partial u(\fat{x})}{\partial n_x}
	\label{eqn:ZZZ_06}
\end{equation}
ここで,$\left|\fat{y}\right|\gg \left| \fat{x} \right|$として近軸近似(paraxial approximation)
\begin{equation}
	r\sim  |\fat{y}|  -\hat{\fat{y}}\cdot \fat{x}, \ \ ( |\fat{y}|\rightarrow \infty)
	\label{eqn:ZZZ_para}
\end{equation}
を用いると,グリーン関数とその法線微分を
\begin{eqnarray}
	G(\fat{x},\fat{y}) &\sim&
	\frac{e^{ik|\fat{y}|}}{4\pi |\fat{y}|} 
	e^{-ik\hat{\fat{y}}\cdot \fat{x}}
	\label{eqn:ZZZ_Gfar}
	\\ 
	H(\fat{x},\fat{y}) &\sim&
	-ik \left( \hat{\fat{y}}\cdot\fat{n}  \right)
	\frac{e^{ik|\fat{y}|}}{4\pi |\fat{y}|} 
	e^{-ik\hat{\fat{y}}\cdot \fat{x}}
	\label{eqn:ZZZ_Hfar}
\end{eqnarray}
と近似できる.これらを式(\ref{eqn:ZZZ_04})に代入すれば,
\begin{equation}
	u^{\rm sc}(\fat{y}) \sim \frac{e^{ik|\fat{y}|}}{4\pi |\fat{y}|} u^{\infty}(k\hat{\fat{y}}), \ \ 
	(|\fat{y}|\rightarrow \infty)
	\label{eqn:ZZZ_07}
\end{equation}
ただし
\begin{equation}
	u^{\infty}(k\hat{\fat{y}}) 
	:=
	\int_{\partial D}
	\left\{t(\fat{x})-ik \left( \hat{\fat{y}}\cdot \fat{n}\right) u(\fat{x}) \right\}
	e^{-ik\hat{\fat{y}}\cdot \fat{x}}
	dS_x
	\label{eqn:ZZZ_Ufar}
\end{equation}
と整理できる.式(\ref{eqn:ZZZ_Ufar})は,ディリクレ問題とノイマン問題,それぞれの
場合に対し次のようになる.
\begin{equation}
	u^{\infty}(k\hat{\fat{y}})  
	= \int _{\partial D} t(\fat{x})e^{-ik\hat{\fat{y}}\cdot \fat{x}} dS_x
	, \ \ ( {\rm Dirichlet})
	\label{eqn:ZZZ_08}
\end{equation}
\begin{equation}
	u^{\infty}(k\hat{\fat{y}})  =
	\int _{\partial D} 
	-ik\left(\hat{\fat{y}}\cdot \fat{n}\right) 
	u(\fat{x})e^{-ik\hat{\fat{y}}\cdot \fat{x}} dS_x
	, \ \ ({\rm Neumann})
	\label{eqn:ZZZ_09}
\end{equation}
ここで,任意の場の量$f(\fat{x})$に対して次のように作用し,
散乱体表面$\partial D$上の積分値を与える特異関数$\gamma_D(\fat{x})$
を導入する.
\begin{equation}
	\int \gamma_D(\fat{x})f(\fat{x})d^3\fat{x} =\int _{\partial D} f(\fat{x}) dS
	\label{eqn:ZZZ_10}
\end{equation}
$\gamma_D(\fat{x})$を用いて式(\ref{eqn:ZZZ_08})と(\ref{eqn:ZZZ_09})を書けば,
積分範囲が全空間となり,$\fat{x}$に関する次のようなフーリエ積分の形式に帰着される.
\begin{equation}
	u^{\infty}(k\hat{\fat{y}})  = \int \gamma_D(\fat{x})t(\fat{x})e^{-ik\hat{\fat{y}}\cdot \fat{x}} dS_x
	, \ \ ( {\rm Dirichlet})
	\label{eqn:ZZZ_11}
\end{equation}
\begin{equation}
	u^{\infty}(k\hat{\fat{y}})  =-\int 
	ik\left(\hat{\fat{y}}\cdot \fat{n}\right) 
	\gamma_D(\fat{x})u(\fat{x})e^{-ik\hat{\fat{y}}\cdot \fat{x}} dS_x
	, \ \ ({\rm Neumann})
	\label{eqn:ZZZ_12}
\end{equation}
\subsubsection{体積散乱体(介在物)}
領域$D$を介在物が占める場合,波数$k$は$D$内外で異なる値をとる.
従って,波数$k$を位置$\fat{x}$の関数とした,
\begin{equation}
	\nabla^2 u(\fat{x}) + k^2(\fat{x}) u(\fat{x}) =0
	\label{eqn:ZZZ_Hlmhlz2}
\end{equation}
がスカラー場$u(\fat{x})$の支配方程式となる.ここでは,介在物外部で波数は
一定値$k_0$であるとし,
\begin{equation}
	n_r(\fat{x}) =\frac{k}{k_0}
	\label{eqn:ZZZ_13}
\end{equation}
および,
\begin{equation}
	V(\fat{x})=n_r^2-1
	\label{eqn:ZZZ_14}
\end{equation}
とおく.このとき,
\begin{equation}
	k^2=k_0^2\left\{ 1 + V(\fat{x}) \right\}
	\label{eqn:ZZZ_15}
\end{equation}
より,式(\ref{eqn:ZZZ_Hlmhlz2})を次のように表すことができる.
\begin{equation}
	\nabla^2 u(\fat{x}) + k_0^2 u(\fat{x}) = -k_0^2 V(\fat{x})u(\fat{x})
	\label{eqn:ZZZ_Hlmhlz3}
\end{equation}
そこで,式(\ref{eqn:ZZZ_Hlmhlz3})の非斉次項を物体力項として扱えば,
散乱場を次のように積分表示することができる.
\begin{equation}
	u^{\rm sc}(\fat{y})
	=u(\fat{y})-u^{in}(\fat{y}) 
	= k_0^2\int G_0(\fat{x},\fat{y})V(\fat{x})u(\fat{x}) d^3\fat{x}
	\label{eqn:ZZZ_LS}
\end{equation}
ただし,$G_0(\fat{x},\fat{y})$は波数$k_0$をもつ均質な無限媒体に対するグリーン関数で,
式(\ref{eqn:ZZZ_Green})において$k=k_0$としたときの解として,
\begin{equation}
	G_0(\fat{x},\fat{y})=\frac{e^{ik_0r}}{4\pi r}, \ \ (r=|\fat{x}-\fat{y}|)
	\label{eqn:ZZZ_G0}
\end{equation}
で与えられる.式(\ref{eqn:ZZZ_LS})はLippman-Schwinger方程式,$V(\fat{x})$は散乱ポテンシャルと呼ばれる.
散乱ポテンシャルは,$\bar D$上でのみゼロでない値をもつので,$V(\fat{x})$を求めることによって
散乱体形状を再構成することができる.ここでも,$G_0(\fat{x},\fat{y})$に近軸近似(\ref{eqn:ZZZ_para})を
用いれば,式(\ref{eqn:ZZZ_LS})は
\begin{equation}
	u^{\rm sc}(\fat{y})
	=
	\frac{e^{ik_0|\fat{y}|}}{4\pi |\fat{y}|}
	\int
	k_0^2 V(\fat{x})u(\fat{x}) e^{-ik\hat{\fat{y}}\cdot \fat{x}}d^3\fat{x}
	\label{eqn:ZZZ_16}
\end{equation}
となり,
\begin{equation}
	u^{\infty}(k\hat{\fat{y}})
	=
	\int
	k_0^2 V(\fat{x})u(\fat{x}) e^{-ik\hat{\fat{y}}\cdot \fat{x}}d^3\fat{x}
	\label{eqn:ZZZ_U_vol}
\end{equation}
とすれば,散乱場を表面散乱体の場合と同様,式(\ref{eqn:ZZZ_07})のような形で表すことができる.
%
\subsection{線形化逆散乱解析}
式(\ref{eqn:ZZZ_11}),式(\ref{eqn:ZZZ_12})および式(\ref{eqn:ZZZ_U_vol})では,
$u^{\infty}(k\hat{\fat{y}})$が式(\ref{eqn:ZZZ_07})によって観測量$u^{sc}(\fat{y})$と結び付けられている.
従って,これらの式の左辺を観測波形から与え,右辺の$\gamma(\fat{x})$や$V(\fat{x})$について解くこと
ができれば,散乱体形状を再構成することができる.
しかしながら,右辺には$u(\fat{x})$や$t(\fat{x})$として未知の場の量が含まれている.
$u(\fat{x})$や$t(\fat{x})$は,求めるべき$\gamma(\fat{x})$や$V(\fat{x})$に
依存し,式(\ref{eqn:ZZZ_11}),(\ref{eqn:ZZZ_12})および(\ref{eqn:ZZZ_U_vol})は
非線形方程式になっている.そのため,このままの形で解くには反復解法を用いる
必要があり一般に計算コストが高い.そこで,$u(\fat{x})$や$t(\fat{x})$を既知量で
近似して方程式を線形化し,反復法を用いることなくフーリエ変換で
散乱体形状の再構成を行う.以下,そのための手順を体積散乱体に対する結果を
式(\ref{eqn:ZZZ_U_vol})から導き,表面散乱体についても同様にして再構成公式が得られることを示す.
\subsubsection{体積散乱体}
$u^{\infty}(k\hat{\fat{y}})$は入射場$u^{in}(\fat{x})$に依存する.
ここでは,次のような平面波が散乱体に入射する場合について考える.
\begin{equation}
	u^{in}(\fat{x})= e^{ik_0\hat{\fat{p}}\cdot\fat{x}}
	\label{eqn:ZZZ_17}
\end{equation}
なお,$\hat{\fat{p}}$は入射平面波の伝播方向を表す単位ベクトルを意味する.
このときの$u^{\infty}(k\hat{\fat{y}})$を,特に,
\begin{equation}
	u^{sc}(k\hat{\fat{y}})
	=
	u^{\infty}(\hat{\fat{y}},\omega;\hat{\fat{p}})
	\label{eqn:ZZZ_18}
\end{equation}
と表し,入射場との関係を明示する.式(\ref{eqn:ZZZ_18})は散乱振幅(scattering amplitude)
と呼ばれ,$\hat{\fat{p}}$方向から入射された平面波によって励起される散乱場の放射パターンを表す.
散乱振幅に関しても,必要の無い限り角周波数$\omega$は省略し,単に
$u^{\infty}(\hat{\fat{y}};\hat{\fat{p}})$と書き,$\hat{\fat{p}}$を送信方向,$\hat{\fat{y}}$を
観測方向と呼ぶ.

ここで,式(\ref{eqn:ZZZ_U_vol})にボルン近似を用い,$u(\fat{x})\simeq u^{in}(\fat{x})$
とすれば,散乱振幅は
\begin{equation}
	u^{\infty}(\hat{\fat{y}};\hat{\fat{p}})
	=
	\int
	k_0^2 V(\fat{x}) e^{-ik(\hat{\fat{y}}-\hat{\fat{p}})\cdot \fat{x}}d^3\fat{x}
	\label{eqn:ZZZ_U_Born}
\end{equation}
と書ける.そこで,散乱ポテンシャル$V(\fat{x})$のフーリエ変換を
\begin{equation}
	\tilde V(\fat{k}) = \int V(\fat{x}) e^{i\fat{k}\cdot \fat{x}} d^3\fat{x}
	\label{eqn:ZZZ_Vk}
\end{equation}
と定義すれば,式(\ref{eqn:ZZZ_U_Born})は
\begin{equation}
	u^{\infty}(\hat{\fat{y}};\hat{\fat{p}}) =k_0^2 \tilde V (
	k(\hat{\fat{y}}-\hat{\fat{p}} )
	)
	\label{eqn:ZZZ_19}
\end{equation}
となる.これは,散乱振幅$u^{\infty}(\hat{\fat{y}};\hat{\fat{p}})$が,散乱ポテンシャル$V(\fat{x})$の
波数ベクトル$k(\hat{\fat{y}}-\fat{x})$におけるスペクトル成分を与えることを意味する.
従って,送受信条件や周波数帯域を適切に設定し,十分な範囲と密度で散乱振幅を
得ることができれば,逆フーリエ変換
\begin{equation}
	V(\fat{x}) =\frac{1}{(2\pi)^3} \int k_0^{-2} u^{\infty}(\hat{\fat{y}};\hat{\fat{p}}) 
	e^{-i\fat{k}\cdot \fat{x}}d^3\fat{k}
	\label{eqn:ZZZ_Vx}
\end{equation}
によって散乱ポテンシャル$V(\fat{x})$が再構成できる.
式(\ref{eqn:ZZZ_Vx})右辺は全波数空間における積分である.
一方,散乱振幅から得られるものは, 式(\ref{eqn:ZZZ_19})から明らかな通り,
\begin{equation}
	\fat{k}=k(\hat{\fat{y}}-\hat{\fat{p}})
	\label{eqn:ZZZ_20}
\end{equation}
における$V(\fat{x})$の波数スペクトル成分である.
従って,式(\ref{eqn:ZZZ_20})のフーリエ積分を正確に評価するためには,
$k,\hat{\fat{y}},\hat{\fat{p}}$を様々に変化させ,できるだけ広い範囲かつ
十分な密度で,$\tilde V(\fat{k})$すなわち$u^{\infty}(\hat{\fat{y}};\hat{\fat{p}})$
を得る必要がある.
そこで,$k$と$\hat{\fat{p}}$を固定し,観測方向$\hat{\fat{y}}$だけを自由に取る
ことができたとする.この場合,式(\ref{eqn:ZZZ_20})の波数ベクトル$\fat{k}$は,
波数空間において,図\ref{fig:ZZZ_Evald}の実線で示した半径$k$の球を描く.
このような球はEvald球と呼ばれる.
つまり,$\hat{\fat{y}}$だけを変化させて観測した場合,Evald球上での
スペクトル成分$\tilde V{\fat{k}}$が得られる.
これだけの情報では,式(\ref{eqn:ZZZ_Vx})のフーリエ積分を評価するために
明らかに十分ではない.そこで,観測方向$\hat{\fat{y}}$に加え,
入射方向$\hat{\fat{p}}$も自由に設定できる場合を考えると,Evald球によって
原点を中心として半径$2k$の球領域の表面と内部を覆うことができる.
このような波数空間の球領域はEvald limitting sphere(Evald限界球)と呼ばれる.
全ての送受信方向で観測を行うことができれば,Evald限界球内部の全ての波数
ベクトルに対するスペクトル$\tilde V(\fat{k})$が得られる.このとき,
波数$k$十分に大きければEvald限界球は波数空間の広い範囲を覆い,
単一周波数による観測でも,散乱ポテンシャルの再構成に十分な情報が得られる.
残念ながら超音波探傷試験では,送受信位置に関する制約が厳しく,送信,受信とも
全周方向から行うことはできず,Evald限界球内の一部のデータしか得られない.
一方で,送受信可能な周波数帯域はある程度広くとることができる.
このときには,送受信方向の制限によりEvald球上の一部でのみデータが
得られるが,波数に応じた半径の異なるEvald球が掃く範囲のスペクトル成分が得られる
ことになる.具体的にどのような波数空間の領域がカバーされるかは,
送受信方向の制約条件によって異なるが,個々のケースでその領域を
図\ref{fig:ZZZ_Evald}のように描画を行って調べることは容易であろう.
\begin{figure}[h]
	\begin{center}
	%\includegraphics[width=0.5\linewidth]{Figs/setup.eps} 
	\includegraphics[width=0.5\linewidth]{Figs/setup.pdf} 
	\end{center}
	\caption{無限領域におけるスカラー波の散乱問題.} 
	\label{fig:ZZZ_100}
\end{figure}
\begin{figure}[h]
	\begin{center}
	%\includegraphics[width=0.6\linewidth]{Figs/Evald.eps} 
	\includegraphics[width=0.6\linewidth]{Figs/Evald.pdf} 
	\end{center}
	\caption{波数ベクトル空間におけるEvaldおよびEvald limitting sphere.} 
	\label{fig:ZZZ_Evald}
\end{figure}
\subsubsection{表面散乱体}
表面散乱体の場合は,散乱場を高周波近似(Kirchhoff近似)を用いて表すことにより,
特異関数$\gamma_D(\fat{x})$に関する再構成式を得ることができる.
ここでも,入射波は式(\ref{eqn:ZZZ_17})の平面波とすれば,式(\ref{eqn:ZZZ_08})
と式(\ref{eqn:ZZZ_09})は,散乱振幅$u^{\infty}(\hat{\fat{y}};\hat{\fat{p}})$
を与える.Kirchhoff近似では,散乱体表面おける散乱波成分を,平面波の反射
係数$R$を用いて,
\begin{equation}
	\left. u^{sc}\right|_{\partial D} \simeq \left. R u^{in}\right|_{\partial D}
	,\\
	\left. t^{sc}\right|_{\partial D} \simeq \left. R t^{in}\right|_{\partial D}
	\label{eqn:ZZZ_Kirch}
\end{equation}
と表す.ただし,$t^{in}$と$t^{sc}$は$\frac{\partial u}{\partial n}$の
入射および散乱波成分を意味し,
\begin{equation}
	t^{in}(\fat{x}) = ik\hat{\fat{p}}\cdot \hat{\fat{n}} e^{ik\hat{\fat{p}}\cdot \hat{\fat{x}}}
	\label{eqn:ZZZ_tin}
\end{equation}
である.
\begin{equation}
	R=\left\{
	\begin{array}{cc}
		-1 & ({\rm Dirichlet}) \\
		 1 & ({\rm Neumann})
	\end{array}
	\right.
	\label{eqn:ZZZ_Rcoef}
\end{equation}
\begin{equation}
	\fat{n}^*= \frac{
		\hat{\fat{p}}- \hat{\fat{y}}
		}{
		\left|
		\hat{\fat{p}}- \hat{\fat{y}}
		\right|
		}
	\label{eqn:}
\end{equation}
\begin{equation}
	u^{\infty}(\hat{\fat{y}};\hat{\fat{p}})
	\simeq
	2ik \left(\hat{\fat{p}}\cdot \fat{n}^*\right)
	\int_{ \partial_D }
	e^{ik( \hat{\fat{p}} - \hat{\fat{y}})\cdot\fat{x}}
	, \ \ ({\rm Dirichlet})
	\label{eqn:ZZZ_uinf_D}
\end{equation}
\begin{equation}
	u^{\infty}(\hat{\fat{y}};\hat{\fat{p}})
	\simeq
	-2ik \left(\hat{\fat{y}}\cdot \fat{n}^*\right)
	\int_{ \partial_D }
	e^{ik( \hat{\fat{p}} - \hat{\fat{y}})\cdot\fat{x}}
	, \ \ ({\rm Neumann})
	\label{eqn:ZZZ_uinf_N}
\end{equation}
$\fat{k}=k(\hat{\fat{p}}-\hat{\fat{y}})$を波数ベクトル,
$\gamma_D(\fat{x})$のフーリエ変換を$\tilde{\gamma}_D(\fat{k})$
\begin{equation}
	\tilde{\gamma}_D(\fat{k}) = 
	\frac{
		u^{\infty}(\hat{\fat{y}};\hat{\fat{p}})
	}{
		2ik\left(\hat{\fat{p}}\cdot \fat{n}^*\right)
	}
	\label{eqn:ZZZ_gmmk_D}
\end{equation}
\begin{equation}
	\tilde{\gamma}_D(\fat{k}) = 
	\frac{
		u^{\infty}(\hat{\fat{y}};\hat{\fat{p}})
	}{
		-2ik\left(\hat{\fat{y}}\cdot \fat{n}^*\right)
	}
	\label{eqn:ZZZ_gmmk_N}
\end{equation}
\begin{equation}
	\gamma_D(\fat{x}) = 
	\frac{1}{(2\pi)^3}\int
	\frac{
		u^{\infty}(\hat{\fat{y}};\hat{\fat{p}})
	}{
		2ik\left(\hat{\fat{p}}\cdot \fat{n}^*\right)
	}e^{-i\fat{k}\cdot\fat{x}}
	d^3\fat{k}, \ \ ({\rm Dirichlet})
	\label{eqn:ZZZ_gmmx_D}
\end{equation}
\begin{equation}
	\gamma_D(\fat{x}) = 
	\frac{1}{(2\pi)^3}\int
	\frac{
		u^{\infty}(\hat{\fat{y}};\hat{\fat{p}})
	}{
		-2ik\left(\hat{\fat{y}}\cdot \fat{n}^*\right)
	}e^{-i\fat{k}\cdot\fat{x}}
	d^3\fat{k}, \ \ ({\rm Neumann})
	\label{eqn:ZZZ_gmmx_N}
\end{equation}
%%%%%%%%%%%%%%%%%%%%%%%%%%%%%%%%%%%%%%%%%%%%%
\subsection{時間領域における再構成式}
式(\ref{eqn:ZZZ_Vx})で与えられる再構成公式が観測波形に対してどのような
処理を行うことを意味するか調べる.そのために,散乱振幅を
\begin{equation}
	u^{\infty}(\hat{\fat{y}};\hat{\fat{p}}) 
	\simeq 
	4\pi |\fat{y}|e^{-ik|\fat{y}|} u^{sc}(\fat{y})
	\label{eqn:ZZZ_21}
\end{equation}
として,式(\ref{eqn:ZZZ_Vx})を
\begin{equation}
	V(\fat{x}) =\frac{1}{2\pi^2} \int k_0^{-2} |\fat{y}|u^{sc}(\fat{y})
	e^{i\fat{k}\cdot \fat{x}-ik |\fat{y}|}d^3\fat{k}, 
	\ \ (\fat{k}=k(\hat{\fat{y}}-\hat{\fat{p}}))
	\label{eqn:ZZZ_Vx_usc}
\end{equation}
と書き直す.この再構成式において, 周波数$\omega$に関する積分を行って
時間領域波形に関する式にすれば,開口合成法の画像化関数と同じ形式を
持つことが明らかになる.ただし,式(\ref{eqn:ZZZ_Vx})の積分を評価するにあたり,
微小体積要素$d^3\fat{k}$を$k$や$\hat{\fat{y}},\hat{\fat{p}}$で
具体的に表現する必要がある. そのためには,送受信方向についての条件を
設定する必要があることから,ここでは次の二つのケースを考える. 
%波数ベクトル$\fat{k}$は入射波の送信方向$\fat{p}$,散乱波の観測方向$\hat{\fat{y}}$と
%角周波数$\omega$に依存する.従って,波数ベクトルに関するフーリエ変換を観測波形を
%使って評価する際には,観測条件に応じた積分変数を取る必要がある.
%では微小体積要素$d^3\fat{k}$を書き下すことができる.
\begin{enumerate}
\item
	入射(あるいは観測)方向$\hat{\fat{p}}(\hat{\fat{y}})$を固定し,
	観測(あるいは入射)方向$\hat{\fat{y}}(\hat{\fat{p}})$を変化させる場合
		(図\ref{fig:ZZZ_101}-(i)および(i')).
	入射方向,観測方向のいずれの方向を固定した場合も以下の議論に
	実質的な差はないため,ここでは入射方向を固定した場合を取り上げる.
\item
	入射方向と観測方向の成す角度を一定に保って同時に変化させる場合
	(図\ref{fig:ZZZ_101}-(ii)). 入射方向と観測方向を
	$\hat{\fat{p}}=-\hat{\fat{y}}$と反対方向にとれば,
	パルス-エコー法による計測となる.
\end{enumerate}
\subsubsection{
	入射方向$\hat{\fat{p}}$を固定し,観測方向$\hat{\fat{y}}$を変化させる場合
}

	観測方向を表すベクトルを次のように球座標を使って表す.
	\begin{equation}
		\hat{\fat{y}}=(\sin\theta\cos\phi,\sin\theta\sin\phi, \cos\theta)
		\label{eqn:ZZZ_22}
	\end{equation}
	このとき,ヤコビ行列式を計算して微小体積要素を求めると
	\begin{equation}
		d^3\fat{k}=k^2\sin\theta (1-\hat{\fat{y}}\cdot\hat{\fat{p}})dkd\theta d\phi
		\label{eqn:ZZZ_23}
	\end{equation}
	となる.ここで,
	\begin{equation}
		\cos 2\delta=\hat{\fat{y}}\cdot\hat{\fat{p}}
		\label{eqn:ZZZ_24}
	\end{equation}
	とおけば,
	\begin{equation}
		d^3\fat{k}=2k^2\sin\theta \sin^2 \left(\delta\right) dkd\theta d\phi
		\label{eqn:ZZZ_25}
	\end{equation}
\subsubsection{
	入射方向と観測方向の成す角度を一定に保って同時に変化させる場合
}
	波数ベクトル方向の単位ベクトルを
	\begin{equation}
		\hat{\fat{k}}=\frac{\hat{\fat{y}}-\hat{\fat{p}}}
		{\left| \hat{\fat{y}}-\hat{\fat{p}} \right| }
		\label{eqn:ZZZ_26}
	\end{equation}
	とすれば,
	\begin{equation}
		\fat{k}= k\left| \hat{\fat{y}}-\hat{\fat{p}} \right| \hat{\fat{k}} 
		\label{eqn:ZZZ_27}
	\end{equation}
	と書くことができる.ここでも,$\hat{\fat{y}} \cdot \hat{\fat{p}}= \cos 2\delta$とすれば,	
	\begin{equation}
		\fat{k}= 2k \sin \delta\hat{\fat{k}} 
	\label{eqn:ZZZ_28}
	\end{equation}
	と表すことができる.$\delta$は一定だから,$\hat{\fat{k}}$を
	\begin{equation}
		\hat{\fat{k}}=(\sin\theta\cos\phi,\sin\theta\sin\phi, \cos\theta)
		\label{eqn:ZZZ_29}
	\end{equation}
	と球座標で表せば,微小体積要素は
	\begin{equation}
		d^3\fat{k}=2\sin \delta
		k^2 \sin\theta dkd\theta d\phi
		\label{eqn:ZZZ_30}
	\end{equation}
	となる.
\begin{figure}[h]
	\begin{center}
	%\includegraphics[width=0.7\linewidth]{Figs/pulse_echo.eps} 
	\includegraphics[width=0.7\linewidth]{Figs/pulse_echo.pdf} 
	\end{center}
	\caption{典型的な送受信モード.} 
	\label{fig:ZZZ_101}
\end{figure}
パルスエコー法(一探触子法)による計測は,式(\ref{eqn:ZZZ_30})において
$2\delta=\pi$とした特別なケースとして表現される.
\subsubsection{体積散乱体}
以上より,$d^3\fat{k}$は,2つの送受信モードをまとめて,
\begin{equation}
	d^3\fat{k}=k^2\hat{J}(\theta,\phi) dkd\theta d\phi
	\label{eqn:ZZZ_30}
\end{equation}
と表すことができる.ただし,
\begin{equation}
	\hat{J}(\theta,\phi) =
	\left\{
	\begin{array}{cc}
		2\sin\theta \sin^2 \delta  & (\hat{\fat{p}}を固定)\\
		2 \sin\theta \sin \delta & (\delta を固定)
	\end{array}
	\right.
	\label{eqn:ZZZ_Jhat}
\end{equation}
で,$\hat{\fat{p}}$を固定する上のケースでは,$\hat{J}$は$\delta$を通じて$\phi$にも依存する.
%式(\ref{eqn:ZZZ_30})を式(\ref{eqn:ZZZ_Vx_usc})に代入すれば,
%\begin{equation}
%	V(\fat{x}) =\frac{1}{2\pi^2} 
%	\int | \fat{y} | u^{sc}(\fat{y}) e^{i\fat{k}\cdot \fat{x}-iky}
%	\hat{J}(\theta,\phi) dkd\theta d\phi
%	\label{eqn:ZZZ_31}
%\end{equation}
ここで,周波数成分$F(\omega)$の平面波を送信点を$\fat{z}$
から入射したとする.この場合は,入射波は
\begin{equation}
	u^{in}(\fat{x})= F(\omega) e^{ik\hat{\fat{p}}\cdot(\fat{x}-\fat{z})}, 
	\label{eqn:ZZZ_32}
\end{equation}
だから,散乱場は
\begin{equation}
	u^{\rm sc}(\fat{y}) \simeq  
	\frac{e^{ik(|\fat{y}|-\hat{\fat{p}}\cdot \fat{z})}}{4\pi |\fat{y}| }
	F(\omega)
	u^{\infty}(\hat{\fat{y}};\hat{\fat{p}}) 
	\label{eqn:ZZZ_33}
\end{equation}
となる.これを,式(\ref{eqn:ZZZ_Vx})に代入すると,
散乱ポテンシャルと観測データの関係が得られる.
\begin{equation}
	V(\fat{x}) =\frac{1}{2\pi^2} 
	\int \frac{|\fat{y}|u^{sc}(\fat{y})}{F(\omega)}
	e^{i\fat{k}\cdot \fat{x}-iky+ik\hat{\fat{p}}\cdot\fat{z}}
	\hat{J} dkd\phi d\theta	
	\label{eqn:ZZZ_34}
\end{equation}
さらに,$\fat{k}=k(\hat{\fat{y}}-\hat{\fat{p}})$と$k=\omega/c$を用いれば,
\begin{equation}
	V(\fat{x}) =\frac{1}{2\pi^2c} \int \frac{\fat{|y|}u^{sc}(\fat{y})}{F(\omega)}
	e^{-ik(y-\hat{\fat{y}}\cdot\fat{x}+\hat{\fat{p}}\cdot(\fat{x}-\fat{z})) }
	\hat{J} d\omega d\phi d\theta	
	\label{eqn:ZZZ_35}
\end{equation}
を得る.ここで,角周波数$\omega$に関するフーリエ変換
\begin{equation}
	a(\fat{y},\fat{z},t):=\frac{1}{2\pi} \int \frac{u^{sc}(\fat{y})}{F(\omega)}e^{-i\omega t}d\omega
	\label{eqn:ZZZ_36}
\end{equation}
で得られる時間波形$a(\fat{y},\fat{z},t)$を用いると,散乱ポテンシャルを
\begin{equation}
	V(\fat{x}) =\frac{1}{2\pi^2c} \int a\left(\fat{y}, \fat{z}, t_{in}+t_{sc} \right) |\fat{y}| \hat{J} d\theta d\phi
	\label{eqn:ZZZ_Vx_time}
\end{equation}
と書くことができる.ただし,$t_{in}$と$t_{sc}$は,
\begin{equation}
	t_{in}=\frac{\hat{\fat{p}}\cdot \left( \fat{x}-\fat{z}\right)}{c}, \ \ 
	t_{sc}=
	\frac{ |\fat{y}|-\hat{\fat{y}}\cdot \fat{x} }{c} \simeq 
	\frac{\left|\fat{y}-\fat{x}\right|}{c} 
	\label{eqn:ZZZ_37}
\end{equation}
とし,それぞれ,入射波と散乱波の伝播時間を意味する.式(\ref{eqn:ZZZ_Vx_time})を
$\theta$と$\varphi$について離散化すれば,積分は次のような和とみなすことができる.
\begin{equation}
	V(\fat{x}) = 
	\frac{1}{2\pi^2c} 
	\sum_{\fat{y},\fat{z}} 
	a\left(\fat{y}, \fat{z}, t_{in}+t_{sc} \right) 
	|\fat{y}| \hat{J} \Delta \theta \Delta \phi
	\label{eqn:ZZZ_38}
\end{equation}
これは,開口合成法における画像化関数の形式と一致する.
式(\ref{eqn:ZZZ_38})は入射波が平面波の場合だが,入射波が球面波の場合には,
送信点を$\fat{z}$とし,$|\fat{x}|\ll 1$における近軸近似を用い,入射波を
\begin{equation}
	u^{in}(\fat{x})
	=F(\omega)\frac{e^{ik|\fat{x}-\fat{z}|}}{4\pi |\fat{x}-\fat{z}|}
	\simeq 
	F(\omega)
	\frac{e^{ik|\fat{z}|}}{4\pi |\fat{z}|}
	e^{-ik\hat{\fat{z}}\cdot\fat{x}}
	\label{eqn:ZZZ_39}
\end{equation}
と表す.式(\ref{eqn:ZZZ_39})は,進行方向$\hat{\fat{p}}=-\hat{\fat{z}}$への平面波
となっているので,散乱振幅を用いて書くと,
\begin{equation}
	u^{\rm sc}(\fat{y}) \simeq  
	\frac{e^{ik(|\fat{y}|+|\fat{z}|-(\hat{\fat{z}}+\hat{\fat{y}})\cdot \fat{x})}}{16\pi^2 |\fat{y}| |\fat{z}|}
	F(\omega)
	u^{\infty}(\hat{\fat{y}};-\hat{\fat{z}}) 
	\label{eqn:ZZZ_usc_far}
\end{equation}
となる.これを式(\ref{eqn:ZZZ_Vx})へ代入すれば,
\begin{equation}
	V(\fat{x}) =\frac{2}{\pi c} \int \frac{|\fat{y}||\fat{z}|u^{sc}(\fat{y})}{F(\omega)}
	e^{-ik(|\fat{y}|+|\fat{z}|-(\hat{\fat{y}}+\hat{\fat{z}})\cdot\fat{x}) }
	\hat{J} d\omega d\phi d\theta	
	\label{eqn:ZZZ_39}
\end{equation}
が得られ,角周波数$\omega$について積分を行うとともに,$\theta,\varphi$で離散化すれば,
\begin{equation}
	V(\fat{x}) = 
	\frac{2}{\pi c} 
	\sum_{\fat{y},\fat{z}} 
	a\left(\fat{y}, \fat{z}, t_{in}+t_{sc} \right) 
	|\fat{y}| |\fat{z}|\hat{J} \Delta \theta \Delta \phi
	\label{eqn:ZZZ_40}
\end{equation}
となる.ただし,伝播時間$t_{in}$と$t_{sc}$はこの場合
\begin{equation}
	t_{in}=
	\frac{ |\fat{z}|-\hat{\fat{z}}\cdot \fat{x} }{c} \simeq 
	\frac{\left|\fat{z}-\fat{x}\right|}{c} 
	, \ \
	t_{sc}=
	\frac{ |\fat{y}|-\hat{\fat{y}}\cdot \fat{x} }{c} \simeq 
	\frac{\left|\fat{y}-\fat{x}\right|}{c} 
	\label{eqn:ZZZ_41}
\end{equation}
である.
\subsubsection{表面散乱体}
表面散乱体の場合も時間領域における再構成式を,体積散乱体の場合と同様にして導くことができる.
ここでも,位置$fat{z}$におかれた周波数スペクトル$F(\omega)$の点波源からの入射波を用いた
とすれば,散乱波は式(\ref{eqn:ZZZ_usc_far})と表すことができる.ただし,散乱振幅は
式(\ref{eqn:ZZZ_gmmx_D})あるいは式(\ref{eqn:ZZZ_gmmx_N})で与られる.
式(\ref{eqn:ZZZ_usc_far})を$u^{\infty}(\hat{\fat{y};\hat{\fat{p}}})$について解き,
ディリクレ散乱体の再構成式(\ref{eqn:ZZZ_usc_far})に用いれば,
\begin{equation}
	\gamma_D(\fat{x}) = -\frac{1}{\pi c^2} \int 
	i\omega
	\frac{|\fat{y}||\fat{z}|u^{sc}(\fat{y})}
	{F(\omega) (\hat{\fat{p}}\cdot\fat{n}^*) }
	e^{-ik(|\fat{y}|+|\fat{z}|-(\hat{\fat{y}}+\hat{\fat{z}})\cdot\fat{x}) }
	\hat{J} d\omega d\phi d\theta	
	\label{eqn:ZZZ_42}
\end{equation}
となる.時間領域波形を式(\ref{eqn:ZZZ_36})で定義し,式(\ref{eqn:ZZZ_42})の$\omega$に関する
積分を行えば,
\begin{equation}
	\gamma_D(\fat{x}) =\frac{1}{4\pi^2c^2} \int \dot{a}
	\left(\fat{y}, \fat{z}, t_{in}+t_{sc} \right) \frac{|\fat{y}| |\fat{z}|}{\hat{\fat{p}}\cdot\fat{n}^*} 
	\hat{J}
	d\theta d\phi, \ \ ({\rm Dirichlet})
	\label{eqn:ZZZ_43}
\end{equation}
Neumann散乱体の場合は,同じ手順に従って次のように得られる.
\begin{equation}
	\gamma_D(\fat{x}) =\frac{1}{4\pi^2c^2} \int \dot{a}
	\left(\fat{y}, \fat{z}, t_{in}+t_{sc} \right) \frac{|\fat{y}| |\fat{z}|}{-\hat{\fat{y}}\cdot\fat{n}^*} 
	\hat{J}
	d\theta d\phi, \ \ ({\rm Neumann})
	\label{eqn:ZZZ_44}
\end{equation}
以上の式(\ref{eqn:ZZZ_43})と(\ref{eqn:ZZZ_44})において,$t_{in}$と$t_{sc}$は
式(\ref{eqn:ZZZ_41})で表される通りで,再構成式の積分を離散化すれば開口合成法の
形式に帰着されることも体積散乱体の場合と同様である.
