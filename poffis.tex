%%%%%%%%%%%%%%%%%%%%%%%%%%%%%%%%%%%%%%%%%%%%%
\section{線形化逆散乱解析}
特異関数の導入
散乱場の積分表現
散乱場の近似
フーリエ変換の形式に整理
フーリエ逆変換で特異関数=欠陥形状を再構成

再構成する対象が明確
計測される波動場の成り立ちについて明確に記述されている

波動場のモデル(計測モデル)を定式化に組み入れることが比較的容易

適用範囲が明確(、波動場の近似が成り立つ条件)
定式化に合うように計測条件を取ることで最適化できる。
高速フーリエ変換をつかった高速化(元々反復法でない上)

波数-周波数領域で定式化される。
これを時間域に戻すことでSAFTとの関連が判明する。
SAFTの適用範囲や最適化はPOFFISに準じて考えることができるようになる。
SAFTという共通点を介して時間反転法の定式化に計測システムモデルを持ち込む方法も検討できるように成る。


POFFIS<--時間域、送受パターンを制限--> SAFT <---sampling function 一般化/特殊化---> time-refersal 
<--- 最適化理論---> FWI, Topological gradient method


線形化逆散乱解析では,散乱波動場を,散乱体形状を与える特異関数を用いて積分で表現する.
積分表現には,未知関数として特異関数と散乱場が含まれる。
そこで、散乱場の近似表現を用いて特異関数だけを未知量とする方程式を求め,
これをフーリエ変換の形式に帰着させることで,反復法を用いることなく
特異関数について解く.
この方法では、再構成される対象である特異関数の意味が明確であること、
再構成に用いられる観測波形がどのような波動場として利用されているか
も明確である。
このことは,再構成式の適用範囲や、画像化および測定条件の最適化について
議論

以下
1) 散乱場の積分表現(表面散乱体、体積散乱体とも同じ形式)
2) 形状を表現する特異関数の導入
3) 散乱場の近似(遠方近似あるいは散乱振幅を仮定)
4) フーリエ逆変換
5) 時間域における画像化関数

以下、3次元スカラー場について
2次元問題や動弾性問題へも適用可能
詳しくは"波動解析と境界要素法の第**章を参照"


\subsection{散乱場の積分表現}
周波数領域におけるスカラー波の散乱問題を考える.
波動場を$u(\fat{x},\omega)$とし,特に必要の無い限り角周波数$\omega$は省略する.
$u(\fat{x})$は,ヘルムホルツ方程式を満足する.
\begin{equation}
	\nabla^2 u(\fat{x}) + k^2 u(\fat{x}) =0, \ \ (\fat{x} \notin \bar{D})
	\label{eqn:ZZZ_00}
\end{equation}
ここに,$k=\omega/c$は一定の波数を表す.
はじめに,無限領域中に置かれた散乱体$D$は,境界$\partial D$上でディリクレ条件
\begin{equation}
	u(\fat{x}) = 0, \ \ \fat{x}\in \partial D
	\label{eqn:ZZZ_01}
\end{equation}
あるいは,ノイマン条件:
\begin{equation}
	\frac{\partial u}{\partial n} = 0 ,\ \ \fat{x}\in \partial D
	\label{eqn:ZZZ_02}
\end{equation}
を満足する場合について考える.
散乱場を$u^{\rm sc}(\fat{x})$,入射波$u^{in}(\fat{x})$とすると,全波動場は
\begin{equation}
	u(\fat{x})=u^{in}(\fat{x})+ u^{\rm sc}(\fat{x})
	\label{eqn:ZZZ_03}
\end{equation}
と表される.散乱場は放射条件を満足するものとすれば,
$u^{\rm sc}(\fat{x})$は次のように積分表現することができる.
\begin{equation}
	u^{\rm sc}(\fat{y}) = \int _{\partial D} 
	\left\{
		G(\fat{x},\fat{y})t(\fat{x})
	-
		H(\fat{x},\fat{y})u(\fat{x})
	\right\} dS_x
	\label{eqn:ZZZ_04}
\end{equation}
ここで,$G(\fat{x},\fat{y})$は
\begin{equation}
	\nabla^2 G(\fat{x},\fat{y}) + k^2 G(\fat{x},\fat{y}) =-\delta (\fat{x}-\fat{y} )
	\label{eqn:ZZZ_05}
\end{equation}
を満足する無限領域のおけるグリーン関数で,
\begin{equation}
	G(\fat{x},\fat{y}) =\frac{e^{ikr}}{4\pi r}, \ \ r=\left| \fat{x}-\fat{y} \right|
	\label{eqn:ZZZ_Green}
\end{equation}
で与えられる.また,
\begin{equation}
	H(\fat{x},\fat{y}) = \frac{\partial G(\fat{x},\fat{y})}{\partial n(\fat{x})}, \ \ 
	t(\fat{x}) =\frac{\partial u(\fat{x})}{\partial n(\fat{x})}
	\label{eqn:ZZZ_06}
\end{equation}
とした.
ここで,$y=\left|\fat{y}\right|\gg \left| \fat{x} \right|$のとき,幾何光学近似
\begin{equation}
	r\sim  y  -\hat{\fat{y}}\cdot \fat{x}, \ \ (y\rightarrow \infty)
	\label{eqn:ZZZ_para}
\end{equation}
を用いると,散乱場の積分表現は,ディリクレ問題,ノイマン問題,それぞれの場合について
\begin{equation}
	u^{\rm sc}(\fat{y}) \sim \frac{e^{iky}}{4\pi y} U(\hat{\fat{y}};s)  
	\label{eqn:ZZZ_07}
\end{equation}
\begin{equation}
	U(\hat{\fat{y}};s)  = \int _{\partial D} t(\fat{x};s)e^{-ik\hat{\fat{y}}\cdot \fat{x}} dS_x
	, \ \ ( {\rm Dirichlet})
	\label{eqn:ZZZ_08}
\end{equation}
\begin{equation}
	U(\hat{\fat{y}};s)  =-\int _{\partial D} u(\fat{x};s)e^{-ik\hat{\fat{y}}\cdot \fat{x}} dS_x
	, \ \ ({\rm Neumann})
	\label{eqn:ZZZ_09}
\end{equation}
なお,$;$に続く$s$は入射波を指定するパラメータを表し,このように表記すること
入射波に依存した量であることを必要あ場合には明示する.
ここで,任意の$D$と場の量$f(\fat{x})$に対して次のように作用する特異関数$\gamma_D(\fat{x})$
\begin{equation}
	\int \gamma_D(\fat{x})f(\fat{x})d^3\fat{x} =\int _{\partial D} f(\fat{x}) dS
	\label{eqn:ZZZ_10}
\end{equation}
を導入する.これを用いると,散乱振幅を
\begin{equation}
	U(\hat{\fat{y}};s)  = \int \gamma_D(\fat{x})t(\fat{x};s)e^{-ik\hat{\fat{y}}\cdot \fat{x}} dS_x
	, \ \ ( {\rm Dirichlet})
	\label{eqn:ZZZ_11}
\end{equation}
\begin{equation}
	U(\hat{\fat{y}};s)  =-\int \gamma_D(\fat{x})u(\fat{x};s)e^{-ik\hat{\fat{y}}\cdot \fat{x}} dS_x
	, \ \ ({\rm Neumann})
	\label{eqn:ZZZ_12}
\end{equation}
と,フーリエ積分の形式で表すことができる.

続いて,領域$D$を介在物が占める場合について考える.
ここで,介在物の外部で波数は$k_0$で一定とし,内部では一般に$k_0$とは異なる
波数となるとする.
そこで,波数を位置の関数として$k(\fat{x})$とすれば,支配方程式であるHelmholtz方程式は
\begin{equation}
	\nabla^2 u(\fat{x},\omega) + k^2(\fat{x}) u(\fat{x},\omega) =0
	\label{eqn:ZZZ_Hlmhlz2}
\end{equation}
となる.ここで,
\begin{equation}
	n_r(\fat{x}) =\frac{k}{k_0}
	\label{eqn:ZZZ_13}
\end{equation}
\begin{equation}
	V(\fat{x})=n_r^2(\fat{x}-1)
	\label{eqn:ZZZ_14}
\end{equation}
とおけば,
\begin{equation}
	k^2=k_0^2 + k_0^2 V(\fat{x})
	\label{eqn:ZZZ_15}
\end{equation}
より,式(\ref{eqn:ZZZ_Hlmhlz2})は
\begin{equation}
	\nabla^2 u(\fat{x},\omega) + k_0(\fat{x}) u(\fat{x},\omega) = -k_0^2 V(\fat{x})u(\fat{x})
	\label{eqn:ZZZ_Hlmhlz3}
\end{equation}
と,形式上,非斉次のHelmholtz方程式の形に書き直すことができる.
従って,右辺の非斉次項を物体力項として扱えば,散乱場を次のように積分表示することができる.
\begin{equation}
	u^{\rm sc}(\fat{y})=u(\fat{y})-u^{in}(\fat{y}) = k_0^2\int G_0(\fat{x},\fat{y})V(\fat{x})u(\fat{x}) d^3\fat{x}
	\label{eqn:ZZZ_LS}
\end{equation}
ここで$G_0(\fat{x},\fat{y})$は波数$k_0$をもつ均質な無限媒体に対するグリーン関数で,
式(\ref{eqn:ZZZ_Green})において$k=k_0$としたものである.
式(\ref{eqn:ZZZ_LS})はLippman-Schwinger方程式,$V(\fat{x})$は散乱ポテンシャルと呼ばれる.
散乱ポテンシャルは$D$を台に持つため,$V(\fat{x})$を求めることができれば,
散乱体形状が再構成される.
最後に,$G_0(\fat{x},\fat{y})$に幾何光学近似(\ref{eqn:ZZZ_para})を用いれば,
式(\ref{eqn:ZZZ_LS})は
\begin{equation}
	u^{\rm sc}(\fat{y})
	=
	\frac{e^{ik_0y}}{4\pi y}
	\int
	k_0^2 V(\fat{x})u(\fat{x}) e^{-ik\hat{\fat{y}}\cdot \fat{x}}d^3\fat{x}
	\label{eqn:ZZZ_16}
\end{equation}
となるので,
\begin{equation}
	U(\hat{\fat{y}},s)
	=
	\int
	k_0^2 V(\fat{x})u(\fat{x};s) e^{-ik\hat{\fat{y}}\cdot \fat{x}}d^3\fat{x}
	\label{eqn:ZZZ_U_vol}
\end{equation}
とすることで,散乱場を表面散乱体の場合と同じ形式に表すことができる.
%
\subsection{線形化逆散乱解析}
はじめに,式(\ref{eqn:ZZZ_LS})に基づき,散乱ポテンシャル$V(\fat{x}$を推定することで
介在物の形状を再構成する方法を示す.いま,散乱体$D$に平面波:
\begin{equation}
	u^{in}(\fat{x})= e^{ik_0\fat{p}\cdot\fat{x}}
	\label{eqn:ZZZ_17}
\end{equation}
を入射したときの$U(\hat{\fat{y}};s)$を
\begin{equation}
	u^{\infty}(\hat{\fat{y}};\fat{p})=U(\hat{\fat{y}};\fat{p})
	\label{eqn:ZZZ_18}
\end{equation}
と表し,これを散乱振幅と呼ぶ.
ここで,$\fat{p}$は入射平面波の伝播方向を表す単位ベクトルを意味する.
このとき,式(\ref{eqn:ZZZ_U_vol})にボルン近似を用い,$u(\fat{x})\simeq u^{in}(\fat{x})$
とすれば,散乱振幅は
\begin{equation}
	u^{\infty}(\hat{\fat{y}},\fat{p})
	=
	\int
	k_0^2 V(\fat{x}) e^{-ik(\hat{\fat{y}}-\fat{p} )\cdot \fat{x}}d^3\fat{x}
	\label{eqn:ZZZ_U_Born}
\end{equation}
と書ける.そこで,散乱ポテンシャル$V(\fat{x})$のフーリエ変換を
\begin{equation}
	\tilde V(\fat{k}) = \int V(\fat{x}) e^{i\fat{k}\cdot \fat{x}} d^3\fat{x}
	\label{eqn:ZZZ_Vk}
\end{equation}
とすれば,式(\ref{eqn:ZZZ_U_Born})は
\begin{equation}
	u^{\infty}(\hat{\fat{y}},\fat{p}) =k_0^2 \tilde V (
	k(\hat{\fat{y}}-\fat{p} )
	)
	\label{eqn:ZZZ_19}
\end{equation}
となることから,散乱振幅が散乱ポテンシャルの波数ベクトル$k(\hat{\fat{y}}-\fat{x})$
でのスペクトル成分であることが分かる.
従って,送受信条件や周波数帯域を適切に設定し,十分な範囲と密度で波数スペクトル成分を
得ることができれば,逆フーリエ変換
\begin{equation}
	V(\fat{x}) =\frac{1}{(2\pi)^3} \int k_0^{-2} u^{\infty}(\hat{\fat{y}};\fat{p}) 
	e^{i\fat{k}\cdot \fat{x}}d^3\fat{k}
	\label{eqn:ZZZ_Vx}
\end{equation}
によって散乱ポテンシャル$V(\fat{x})$を再構成できる.
ここに,式(\ref{eqn:ZZZ_Vx})における波数ベクトル$\fat{k}$は
\begin{equation}
	\fat{k}=k(\hat{\fat{y}}-\fat{p})
	\label{eqn:ZZZ_20}
\end{equation}
だから,フーリエ積分の評価に用いることのできる波数ベクトル成分は,角周波数$\omega$,
観測方向$\hat{\fat{y}}$,入射方向$\fat{p}$に依って決まる.
いま,$\omega$と$\fat{p}$を固定し,観測方向だけを自由に取ることができる場合を考えると,
$\fat{k}$は波数空間で,図\ref{fig:ZZZ_Evald}に実線で示したような半径$k$球を描き,
これをEvald球と呼ぶ.つまり,観測方向だけを変化させて観測した場合,
散乱ポテンシャルのEvald球状の成分だけしか得られない.一方,観測方向$\hat{\fat{y}}$に加え,
入射方向$\fat{p}$も任意に選べる場合,Evald球は原点を中心として半径$2k$の球表面と内部を履く.
このような波数空間の球領域はEvald limitting sphere(Evald限界球)と呼ばれる.
全ての送受信方向で観測を行うことができれば,Evald限界球内部の
波数成分全ての情報が得られるため,波数$k$十分に大きければ,単一の波数成分
のデータだけで,散乱ポテンシャルの十分な情報が得られる.
しかしながら,例えば超音波探傷試験では,送受信方向は制約が厳しく,
ごく限られた方向からの送受信しか行うことができないために,Evald限界球内の
一部のデータしか得られない.
一方で,送受信することのできる周波数帯域はパルスやチャープ波を用いることで
ある程度広くとることができる.このときには,波数$k$が$k\in \left(k_{min},k_{max} \right)$
の範囲にあるEvald球の一部が掃く領域が単一周波数(monochromatic)での
検査に比べて広くとることができ,散乱ポテンシャルに関するより多くの情報を得ることができる.
\\

ここで,散乱振幅を
\begin{equation}
	u^{\infty}(\hat{\fat{y}};\fat{p})=4\pi ye^{-iky} u^{sc}(\fat{y})
	\label{eqn:ZZZ_21}
\end{equation}
として,観測波形データ$u^{sc}$を用いて式(\ref{eqn:ZZZ_Vx})を書き直すと
\begin{equation}
	V(\fat{x}) =\frac{1}{2\pi^2} \int k_0^{-2} yu^{sc}(\fat{y})
	e^{i\fat{k}\cdot \fat{x}-iky}d^3\fat{k}, 
	\ \ (\fat{k}=k(\hat{\fat{y}}-\fat{p}))
	\label{eqn:ZZZ_Vx_usc}
\end{equation}
となる.この再構成式がどのような操作を意味するかは,次節で述べるように,
周波数$\omega$に関する積分を行い時間領域における再構成式に直すことで
明らかになる.
\begin{figure}[h]
	\begin{center}
	%\includegraphics[width=0.5\linewidth]{Figs/setup.eps} 
	\includegraphics[width=0.5\linewidth]{Figs/setup.pdf} 
	\end{center}
	\caption{無限領域におけるスカラー波の散乱問題.} 
	\label{fig:ZZZ_100}
\end{figure}
\begin{figure}[h]
	\begin{center}
	%\includegraphics[width=0.6\linewidth]{Figs/Evald.eps} 
	\includegraphics[width=0.6\linewidth]{Figs/Evald.pdf} 
	\end{center}
	\caption{波数ベクトル空間におけるEvaldおよびEvald limitting sphere.} 
	\label{fig:ZZZ_Evald}
\end{figure}
\subsection{時間領域における再構成式}
式(\ref{eqn:ZZZ_Vx})の積分を評価するにあたり,微小体積要素$d^3\fat{k}$を 
$\fat{p}$,$\hat{\fat{y}}$, $k$を用いて表す必要がある.
%波数ベクトル$\fat{k}$は入射波の送信方向$\fat{p}$,散乱波の観測方向$\hat{\fat{y}}$と
%角周波数$\omega$に依存する.従って,波数ベクトルに関するフーリエ変換を観測波形を
%使って評価する際には,観測条件に応じた積分変数を取る必要がある.
ここでは,次のような二つのケースについて考える.
%では微小体積要素$d^3\fat{k}$を書き下すことができる.
\begin{enumerate}
\item
	送信(受信)方向$\fat{p}(\hat{\fat{y}})$を固定し,
	受信(送信)方向$\hat{\fat{y}}(\fat{p})$を変化させる場合\\

	観測方向を表すベクトルを次のように球座標を使って表す.
	\begin{equation}
		\hat{\fat{y}}=(\sin\theta\cos\phi,\sin\theta\sin\phi, \cos\theta)
		\label{eqn:ZZZ_22}
	\end{equation}
	このとき,微小体積要素はヤコビ行列式を計算すれば
	\begin{equation}
		d^3\fat{k}=k^2\sin\theta (1-\hat{\fat{y}}\cdot\fat{p})dkd\theta d\phi
		\label{eqn:ZZZ_23}
	\end{equation}
	となることが示される.
	ここで,
	\begin{equation}
		\cos\delta=\hat{\fat{y}}\cdot\fat{p}
		\label{eqn:ZZZ_24}
	\end{equation}
	とおけば,
	\begin{equation}
		d^3\fat{k}=4k^2\sin\theta \sin^2 \frac{\delta}{2} dkd\theta d\phi
		\label{eqn:ZZZ_25}
	\end{equation}
\item
	送信方向と受信方向が成す角度を一定に保ち,送受方向を同時に変化させる場合\\

	単位ベクトル:
	\begin{equation}
		\hat{\fat{k}}=\frac{\hat{\fat{y}}-\fat{p}}
		{\left| \fat{y}-\fat{p} \right| }
		\label{eqn:ZZZ_26}
	\end{equation}
	を用いれば,波数ベクトル$\fat{k}$は
	\begin{equation}
		\fat{k}= k\left| \hat{\fat{y}}-\fat{p} \right| \hat{\fat{k}} 
		\label{eqn:ZZZ_27}
	\end{equation}
	と書くことができ,$\hat{\fat{y}} \cdot \fat{p}= \cos \delta$とおけば,	
	\begin{equation}
		\fat{k}= 2k \sin \frac{\delta}{2}\hat{\fat{k}} 
	\label{eqn:ZZZ_28}
	\end{equation}
	と表すことができる.$\delta$は一定だから,$\hat{\fat{k}}$を
	\begin{equation}
		\hat{\fat{k}}=(\sin\theta\cos\phi,\sin\theta\sin\phi, \cos\theta)
		\label{eqn:ZZZ_29}
	\end{equation}
	と球座標を使って表せば,微小体積要素を
	\begin{equation}
		d^3\fat{k}=2\sin\frac{\delta}{2} 
		k^2 \sin\theta dkd\theta d\phi
		\label{eqn:ZZZ_30}
	\end{equation}
\end{enumerate}
\begin{figure}[h]
	\begin{center}
	%\includegraphics[width=0.7\linewidth]{Figs/pulse_echo.eps} 
	\includegraphics[width=0.7\linewidth]{Figs/pulse_echo.pdf} 
	\end{center}
	\caption{送受信パターン.} 
	\label{fig:ZZZ_101}
\end{figure}
パルスエコー法(一探触子法)による計測は後者の特別な場合に相当する.
以上より,$d^3\fat{k}$は
\begin{equation}
	d^3\fat{k}=k^2\hat{J}(\theta,\phi) dkd\theta d\phi
	\label{eqn:ZZZ_30}
\end{equation}
と表すことができる.
これを式(\ref{eqn:ZZZ_Vx_usc})に代入することにより次の再構成式が得られる.
\begin{equation}
	V(\fat{x}) =\frac{1}{2\pi^2} \int yu^{sc}(\fat{y})
	e^{i\fat{k}\cdot \fat{x}-iky}
	\hat{J}(\theta,\phi) dkd\theta d\phi
	\label{eqn:ZZZ_31}
\end{equation}

ここで,送信点を$\fat{z}$とし,
\begin{equation}
	u^{in}(\fat{x})= F(\omega) e^{ik_0\fat{p}\cdot(\fat{x}-\fat{z})}
	\label{eqn:ZZZ_32}
\end{equation}
のような周波数成分$F(\omega)$をもつ平面波を入射する場合について考える.
このとき,散乱場は
\begin{equation}
	u^{\rm sc}(\fat{y})= \frac{e^{ik_0(y-\fat{p}\cdot \fat{z})}}{4\pi y }F(\omega)U(\hat{\fat{y}};\fat{p}) 
	\label{eqn:ZZZ_33}
\end{equation}
と書くことができる.よって,散乱ポテンシャルは観測データとして得られる
散乱場を使って次のように表される.
\begin{equation}
	V(\fat{x}) =\frac{1}{2\pi^2} \int \frac{u^{sc}(\fat{y})}{F(\omega)}
	e^{i\fat{k}\cdot \fat{x}-iky+ik\fat{p}\cdot\fat{z}}
	\hat{J} dkd\phi d\theta	
	\label{eqn:ZZZ_34}
\end{equation}
さらに,$\fat{k}=k(\hat{\fat{y}}-\fat{p})$を代入し,$k=\omega/c$を用いれば,
\begin{equation}
	V(\fat{x}) =\frac{1}{2\pi^2c} \int \frac{yu^{sc}(\fat{y})}{F(\omega)}
	e^{-ik_0(y-\hat{\fat{y}}\cdot\fat{x}+\fat{p}\cdot(\fat{x}-\fat{z})) }
	\hat{J} d\omega d\phi d\theta	
	\label{eqn:ZZZ_35}
\end{equation}
を得る.
ここで,角周波数$\omega$に関するフーリエ変換
\begin{equation}
	a(\fat{y},\fat{z},t):=\frac{1}{2\pi} \int \frac{u^{sc}(\fat{y})}{F(\omega)}e^{-i\omega t}d\omega
	\label{eqn:ZZZ_36}
\end{equation}
で得られる時間波形を用いると,散乱ポテンシャルを
\begin{equation}
	V(\fat{x}) =\frac{1}{2\pi^2c} \int a\left(\fat{y}, \fat{z}, t_{in}+t_{sc} \right) y \hat{J} d\theta d\phi
	\label{eqn:ZZZ_Vx_time}
\end{equation}
と表される.ただし,$t_{in},t_{sc}$はそれぞれ
\begin{equation}
	t_{in}=\frac{\fat{p}\cdot \left( \fat{x}-\fat{z}\right)}{c}, \ \ 
	t_{sc}=
	\frac{ y-\hat{\fat{y}}\cdot \fat{x} }{c} \simeq 
	\frac{\left|\fat{y}-\fat{x}\right|}{c} 
	\label{eqn:ZZZ_37}
\end{equation}
で,入射波と散乱波の伝播時間を意味する.
式(\ref{eqn:ZZZ_Vx_time})を離散的な$\fat{y},\fat{z}$について評価する場合,
積分は次のような和とみなすことができ,これは,発見的に導かれた開口合成法の
形式と一致する.
\begin{equation}
	V(\fat{x}) = 
	\frac{1}{2\pi^2c} 
	\sum_{\fat{y},\fat{z}} 
	a\left(\fat{y}, \fat{z}, t_{in}+t_{sc} \right) y \hat{J} \Delta \theta \Delta \phi
	\label{eqn:ZZZ_38}
\end{equation}

